%                                                                 aa.dem
% AA vers. 9.1, LaTeX class for Astronomy & Astrophysics
% demonstration file
%                                                       (c) EDP Sciences
%-----------------------------------------------------------------------
%
%\documentclass[referee]{aa} % for a referee version
%\documentclass[onecolumn]{aa} % for a paper on 1 column
%\documentclass[longauth]{aa} % for the long lists of affiliations
%\documentclass[letter]{aa} % for the letters
%\documentclass[bibyear]{aa} % if the references are not structured
%                              according to the author-year natbib style

%
\documentclass[twocolumn]{aa}

%
\usepackage{graphicx}
%%%%%%%%%%%%%%%%%%%%%%%%%%%%%%%%%%%%%%%%
\usepackage{txfonts}
%%%%%%%%%%%%%%%%%%%%%%%%%%%%%%%%%%%%%%%%
%\usepackage[options]{hyperref}
% To add links in your PDF file, use the package "hyperref"
% with options according to your LaTeX or PDFLaTeX drivers.
%
\begin{document}


   \title{Modelling the formation of the GD-1 stellar stream inside a host with a fermionic dark matter core-halo distribution.}

   % \subtitle{I. Overviewing the $\kappa$-mechanism}

   \author{Martín F. Mestre\inst{1,2}\fnmsep\thanks{
         \email{mmestre@fcaglp.unlp.edu.ar}
      }
      \and
      Carlos R. Argüelles\inst{1,2}
      \and
      Daniel D. Carpintero\inst{1,2}
   }

   \institute{Instituto de Astrof{\'i}sica de La Plata (CONICET-UNLP), Paseo del
               Bosque S/N, La Plata (1900), Buenos Aires, Argentina\\
               \and
              Facultad de Ciencias Astron{\'o}micas y Geof{\'i}sicas de La Plata (UNLP),
              Paseo del Bosque S/N, La Plata (1900), Buenos Aires, Argentina\\
             }

   \date{Received February 29, 2023; accepted February 30, 2023}

% \abstract{}{}{}{}{}
% 5 {} token are mandatory

  \abstract
  % context heading (optional)
  % {} leave it empty if necessary
   {Stellar streams are a consequence of the tidal forces produced by a host galaxy on its satellites (i.e. globular clusters and dwarf spheroidals).
   As the self-gravity of stellar streams is almost negligible, they constitute excellent probes of the gravitational potential of the host galaxy. For this reason, some Milky Way stellar streams have been used to put constraints on the dark matter (DM) total mass and shape, under empirical DM distributions (i.e. NFW, logarithmic, etc). Within the set of non-empirical DM distributions, there exists a DM model
   deduced from first principles by means of the maximization of the coarse-grained entropy for self-gravitating fermions, the MEPP, after Maximum Entropy Production Principle. MEPP profiles have a rich
   variety of behaviours, being a large subfamily characterized by a dense core of mass
   $\approx4.1\times10^6 M_\odot$ (that can mimick a black hole regarding the orbits of the S-stars at Sagittario A*) and an extended halo with the plasticity of a Burkert profile.}
  % aims heading (mandatory)
   {In this work we attempt to model the GD-1 stellar stream using a spherical MEPP distribution which, at the same time, is in sintony with previous fits to the S-stars.}
  % methods heading (mandatory)
   {For that purpose we used a genetic algorithm in order to fit both the stream orbit's initial conditions and the fermionic halo. We modelled the barionic potential with a bulge and two disks (thin and thick) with fixed parameters according to the recent literature. The stream observable is 6D phase-space data from the Gaia DR2 survey.}
  % results heading (mandatory)
   {We were able to find good fits for a 1D continuous subfamily of models parametrized by the fermion
   mass going from $56$~keV/$c^2$ to $~360$~keV/$c^2$, respectively corresponding to core radii going from
   $10^3$ to 10 Schwarzschild radii. For smaller and larger values of the fermion mass, there is no solution that simultaneously fits the GD-1 stream and the S-stars. The solutions have a virial radius of 28 kpc and a virial mass of $1.4\times10^{11} M_\odot$, the latter being at $2\sigma$ from previous Milky Way DM halo mass estimates using the Sagittarius stream. We do not assume the velocity of the local standard of rest ($v_\mathrm{lsr}$) and the result gives $v_\mathrm{lsr}\approx244$ km/s, in agreement with recent independent estimates.}
  % conclusions heading (optional), leave it empty if necessary
   {}

   \keywords{Galaxy: kinematics and dynamics --
             Galaxy: stellar streams --
             dark matter
               }
   \titlerunning{The GD-1 stream inside a fermionic dark matter halo.}

   \maketitle

%________________________________________________________________


\section{Introduction}

Stellar streams probe the acceleration field produced by the Milky Way.
This information together with barionic measurements help in making claims about the
dark matter halo.

\section{Methodology}
In this section we explain the observables and methods used in this research.

\subsection{Observables}


\subsection{The fermionic dark matter model}
Our dark matter model consists of a spherical and isotropic distribution of fermions at finite temperature in hydrostatic equilibrium subject to the law of General Relativity (i.e. T.O.V equations) as defined in~\cite{arguelles_novel_2018} with the particularity that we have made a few change of variables in order to make the numerical computation more robust. The system of equations thus obtained is as follows:

\begin{eqnarray}
   \label{sode}
   \frac{d\nu}{d\zeta} & = & \frac{1}{2}\left[e^{z}+e^{2\zeta}\frac{P(r)}{\rho_{rel}c^2}\right][1-e^{z}]^{-1},\\
   \frac{dz}{d\zeta} & = &-1+e^{(2\zeta-z)}\frac{\rho(r)}{\rho_{rel}},
\end{eqnarray}
where
\begin{eqnarray}
   \zeta &=& \ln(r/R),\\
   z &=&\ln\psi,\quad \psi = \frac{M(r)}{M}\frac{R}{r},\\
   \rho(r)&=&\frac{4\rho_{\rm rel}}{\sqrt{\pi}}\int^\infty_1\epsilon^2[\epsilon^2-1]^{1/2}f(r,\epsilon)d\epsilon,\\
   P(r)&=&\frac{4\rho_{\rm rel}c^2}{3\sqrt{\pi}}\int^\infty_1[\epsilon^2-1]^{3/2}f(r,\epsilon)d\epsilon,\\
   R^2 &=& \frac{c^2}{8\pi G \rho_{\rm{rel}}},\\
   M &=& 4\pi R^3 \rho_{\rm{rel}},\\
   \rho_{\rm{rel}}&=& \frac{ G\pi^{3/2} m^4 c^3}{h^3},
\end{eqnarray}
where $c$ is the speed of light, $h$ is the Planck constant, $m$ is the fermion mass, $M(r)$ is the mass enclosed up to radius $r$ and $\nu(r)$ is the metric exponent, i.e. we use the convention $g_{00}=e^{\nu(r)}$. The phase-space function ($f$) is given by a Fermi-Dirac distribution with energy cutoff:
\begin{equation}
f(r,\epsilon)=
   \begin{cases}
      \frac{1-e^{[\epsilon-\epsilon_c(r)]/\beta(r)}}
      {1+e^{[\epsilon-\alpha(r)]/\beta(r)}}\quad \epsilon \leq \epsilon_c(r)\\
      0\quad \epsilon > \epsilon_c(r),
   \end{cases}
\end{equation}
where $\alpha(r)$ is the chemical potential (including rest mass), $\epsilon(r)$ is the cutoff energy(including rest mass) and $\beta(r)$ is the temperature variable.

The above equations should be complemented with two thermodinamic equilibrium conditions given
in~\citet{PhysRev.35.904} and~\citet{RevModPhys.21.531} together with the condition of energy conservation
along the geodesic given in~\citet{1989A&A...221....4M}:
\begin{equation}
   \frac{1}{\alpha}\frac{d\alpha}{dr}=\frac{1}{\beta}\frac{d\beta}{dr}=
   \frac{1}{\epsilon}\frac{d\epsilon}{dr}=-\frac{1}{2}\frac{d\nu}{dr}.
   \label{geod_energy}
\end{equation}

It is not possible to integrate this equations from $r=0$ because the right hand the change of variables $\zeta(r)$ is divergent at the origin. Nevertheless, it is possible
to reach the following approximations for the initial conditions at a value $r_{\rm{min}}\gtrsim 0$:
\begin{eqnarray}
   \nu(r_{\rm{min}})-\nu_0 &\approx& \frac{1}{6}\frac{\rho_0}{\rho_{\rm{rel}}}\left[\frac{r_{\rm{min}}}{R}\right]^2\equiv\tau, \\
   \psi(r_{\rm{min}})&\approx&\frac{1}{3}\frac{\rho_0}{\rho_{\rm{rel}}}\left[\frac{r_{\rm{min}}}{R}\right]^2=2\tau,
\end{eqnarray}
which implies
\begin{equation}
   \frac{r_{\rm{min}}}{R}=\sqrt{6\tau\frac{\rho_{\rm{rel}}}{\rho_0}},
\end{equation}
where $rho_0\equiv \rho(0)$. The initial condition of the metric, $\nu_0=\nu(0)$, is determined
afterwards with a condition of continuity with the Schwarzschild metric at the border of the
fermion distribution:
$$2\ln\left(\frac{\beta_b}{\beta_0}\sqrt{1-\psi_b}\right)$$,
where $\psi_b$ and $\beta_b$ are quantities evaluated at the border.

The system of equations thus obtained has four free parameters: $m$, $\alpha_0=\alpha(0)$,
$\beta_0=\beta(0)$, and $\theta_0=\theta(0)$ but in practice we will use the following related
quantities as parameters:


Equations~\ref{sode}-\ref{geod_energy} are solved using a {\scshape{Python}}~\citep{van1995python} script\footnote{\url{https://github.com/martinmestre/stream-fit/blob/main/likelihood_grid/optim/model_def.py}} that makes use of the {\scshape{NumPy}}~\citep{2020SciPy-NMeth} and {\scshape{SciPy}}~\citep{harris2020array} packages.




\subsection{Milky Way model}
\subsection{Stream model}
\subsection{Optimization algorithm}

\section{Results}

\section{Discussion}

\section{Conclusions}


%______________________________________________________________


\begin{acknowledgements}
    Thank to people, institutions and codes used.
\end{acknowledgements}

% WARNING
%-------------------------------------------------------------------
% Please note that we have included the references to the file aa.dem in
% order to compile it, but we ask you to:
%
% - use BibTeX with the regular commands:
%   \bibliographystyle{aa} % style aa.bst
%   \bibliography{Yourfile} % your references Yourfile.bib
%
% - join the .bib files when you upload your source files
%-------------------------------------------------------------------
\bibliographystyle{aa} % style aa.bst
\bibliography{refs} % your references Yourfile.bib
%

\end{document}


%%%%%%%%%%%%%%%%%%%%%%%%%%%%%%%%%%%%%%%%%%%%%%%%%%%%%%%%%%%%%%
Example below of non-structurated natbib references
To use the v8.3 macros with this form of composition of bibliography,
the option "bibyear" should be added to the command line
"\documentclass[bibyear]{aa}".
%%%%%%%%%%%%%%%%%%%%%%%%%%%%%%%%%%%%%%%%%%%%%%%%%%%%%%%%%%%%%%



%%%%%%%%%%%%%%%%%%%%%%%%%%%%%%%%%%%%%%%%%%%%%%%%%%%%%%%%%%%%%%
Examples for figures using graphicx
A guide "Using Imported Graphics in LaTeX2e"  (Keith Reckdahl)
is available on a lot of LaTeX public servers or ctan mirrors.
The file is : epslatex.pdf
%%%%%%%%%%%%%%%%%%%%%%%%%%%%%%%%%%%%%%%%%%%%%%%%%%%%%%%%%%%%%%

%_____________________________________________________________
%                 A figure as large as the width of the column
%-------------------------------------------------------------
   \begin{figure}
   \centering
   \includegraphics[width=\hsize]{empty.eps}
      \caption{Vibrational stability equation of state
               $S_{\mathrm{vib}}(\lg e, \lg \rho)$.
               $>0$ means vibrational stability.
              }
         \label{FigVibStab}
   \end{figure}
%
%_____________________________________________________________
%                                    One column rotated figure
%-------------------------------------------------------------
   \begin{figure}
   \centering
   \includegraphics[angle=-90,width=3cm]{empty.eps}
      \caption{Vibrational stability equation of state
               $S_{\mathrm{vib}}(\lg e, \lg \rho)$.
               $>0$ means vibrational stability.
              }
         \label{FigVibStab}
   \end{figure}
%
%_____________________________________________________________
%                        Figure with caption on the right side
%-------------------------------------------------------------
   \begin{figure}
   \sidecaption
   \includegraphics[width=3cm]{empty.eps}
      \caption{Vibrational stability equation of state
               $S_{\mathrm{vib}}(\lg e, \lg \rho)$.
               $>0$ means vibrational stability.
              }
         \label{FigVibStab}
   \end{figure}
%
%_____________________________________________________________
%
%_____________________________________________________________
%                                Figure with a new BoundingBox
%-------------------------------------------------------------
   \begin{figure}
   \centering
   \includegraphics[bb=10 20 100 300,width=3cm,clip]{empty.eps}
      \caption{Vibrational stability equation of state
               $S_{\mathrm{vib}}(\lg e, \lg \rho)$.
               $>0$ means vibrational stability.
              }
         \label{FigVibStab}
   \end{figure}
%
%_____________________________________________________________
%
%_____________________________________________________________
%                                      The "resizebox" command
%-------------------------------------------------------------
   \begin{figure}
   \resizebox{\hsize}{!}
            {\includegraphics[bb=10 20 100 300,clip]{empty.eps}
      \caption{Vibrational stability equation of state
               $S_{\mathrm{vib}}(\lg e, \lg \rho)$.
               $>0$ means vibrational stability.
              }
         \label{FigVibStab}
   \end{figure}
%
%______________________________________________________________
%
%_____________________________________________________________
%                                             Two column Figure
%-------------------------------------------------------------
   \begin{figure*}
   \resizebox{\hsize}{!}
            {\includegraphics[bb=10 20 100 300,clip]{empty.eps}
      \caption{Vibrational stability equation of state
               $S_{\mathrm{vib}}(\lg e, \lg \rho)$.
               $>0$ means vibrational stability.
              }
         \label{FigVibStab}
   \end{figure*}
%
%______________________________________________________________
%
%_____________________________________________________________
%                                             Simple A&A Table
%_____________________________________________________________
%
\begin{table}
\caption{Nonlinear Model Results}             % title of Table
\label{table:1}      % is used to refer this table in the text
\centering                          % used for centering table
\begin{tabular}{c c c c}        % centered columns (4 columns)
\hline\hline                 % inserts double horizontal lines
HJD & $E$ & Method\#2 & Method\#3 \\    % table heading
\hline                        % inserts single horizontal line
   1 & 50 & $-837$ & 970 \\      % inserting body of the table
   2 & 47 & 877    & 230 \\
   3 & 31 & 25     & 415 \\
   4 & 35 & 144    & 2356 \\
   5 & 45 & 300    & 556 \\
\hline                                   %inserts single line
\end{tabular}
\end{table}
%
%_____________________________________________________________
%                                             Two column Table
%_____________________________________________________________
%
\begin{table*}
\caption{Nonlinear Model Results}
\label{table:1}
\centering
\begin{tabular}{c c c c l l l }     % 7 columns
\hline\hline
                      % To combine 4 columns into a single one
HJD & $E$ & Method\#2 & \multicolumn{4}{c}{Method\#3}\\
\hline
   1 & 50 & $-837$ & 970 & 65 & 67 & 78\\
   2 & 47 & 877    & 230 & 567& 55 & 78\\
   3 & 31 & 25     & 415 & 567& 55 & 78\\
   4 & 35 & 144    & 2356& 567& 55 & 78 \\
   5 & 45 & 300    & 556 & 567& 55 & 78\\
\hline
\end{tabular}
\end{table*}
%
%-------------------------------------------------------------
%                                          Table with notes
%-------------------------------------------------------------
%
% A single note
\begin{table}
\caption{\label{t7}Spectral types and photometry for stars in the
  region.}
\centering
\begin{tabular}{lccc}
\hline\hline
Star&Spectral type&RA(J2000)&Dec(J2000)\\
\hline
69           &B1\,V     &09 15 54.046 & $-$50 00 26.67\\
49           &B0.7\,V   &*09 15 54.570& $-$50 00 03.90\\
LS~1267~(86) &O8\,V     &09 15 52.787&11.07\\
24.6         &7.58      &1.37 &0.20\\
\hline
LS~1262      &B0\,V     &09 15 05.17&11.17\\
MO 2-119     &B0.5\,V   &09 15 33.7 &11.74\\
LS~1269      &O8.5\,V   &09 15 56.60&10.85\\
\hline
\end{tabular}
\tablefoot{The top panel shows likely members of Pismis~11. The second
panel contains likely members of Alicante~5. The bottom panel
displays stars outside the clusters.}
\end{table}
%
% More notes
%
\begin{table}
\caption{\label{t7}Spectral types and photometry for stars in the
  region.}
\centering
\begin{tabular}{lccc}
\hline\hline
Star&Spectral type&RA(J2000)&Dec(J2000)\\
\hline
69           &B1\,V     &09 15 54.046 & $-$50 00 26.67\\
49           &B0.7\,V   &*09 15 54.570& $-$50 00 03.90\\
LS~1267~(86) &O8\,V     &09 15 52.787&11.07\tablefootmark{a}\\
24.6         &7.58\tablefootmark{1}&1.37\tablefootmark{a}   &0.20\tablefootmark{a}\\
\hline
LS~1262      &B0\,V     &09 15 05.17&11.17\tablefootmark{b}\\
MO 2-119     &B0.5\,V   &09 15 33.7 &11.74\tablefootmark{c}\\
LS~1269      &O8.5\,V   &09 15 56.60&10.85\tablefootmark{d}\\
\hline
\end{tabular}
\tablefoot{The top panel shows likely members of Pismis~11. The second
panel contains likely members of Alicante~5. The bottom panel
displays stars outside the clusters.\\
\tablefoottext{a}{Photometry for MF13, LS~1267 and HD~80077 from
Dupont et al.}
\tablefoottext{b}{Photometry for LS~1262, LS~1269 from
Durand et al.}
\tablefoottext{c}{Photometry for MO2-119 from
Mathieu et al.}
}
\end{table}
%
%-------------------------------------------------------------
%                                       Table with references
%-------------------------------------------------------------
%
\begin{table*}[h]
 \caption[]{\label{nearbylistaa2}List of nearby SNe used in this work.}
\begin{tabular}{lccc}
 \hline \hline
  SN name &
  Epoch &
 Bands &
  References \\
 &
  (with respect to $B$ maximum) &
 &
 \\ \hline
1981B   & 0 & {\it UBV} & 1\\
1986G   &  $-$3, $-$1, 0, 1, 2 & {\it BV}  & 2\\
1989B   & $-$5, $-$1, 0, 3, 5 & {\it UBVRI}  & 3, 4\\
1990N   & 2, 7 & {\it UBVRI}  & 5\\
1991M   & 3 & {\it VRI}  & 6\\
\hline
\noalign{\smallskip}
\multicolumn{4}{c}{ SNe 91bg-like} \\
\noalign{\smallskip}
\hline
1991bg   & 1, 2 & {\it BVRI}  & 7\\
1999by   & $-$5, $-$4, $-$3, 3, 4, 5 & {\it UBVRI}  & 8\\
\hline
\noalign{\smallskip}
\multicolumn{4}{c}{ SNe 91T-like} \\
\noalign{\smallskip}
\hline
1991T   & $-$3, 0 & {\it UBVRI}  &  9, 10\\
2000cx  & $-$3, $-$2, 0, 1, 5 & {\it UBVRI}  & 11\\ %
\hline
\end{tabular}
\tablebib{(1)~\citet{branch83};
(2) \citet{phillips87}; (3) \citet{barbon90}; (4) \citet{wells94};
(5) \citet{mazzali93}; (6) \citet{gomez98}; (7) \citet{kirshner93};
(8) \citet{patat96}; (9) \citet{salvo01}; (10) \citet{branch03};
(11) \citet{jha99}.
}
\end{table}
%_____________________________________________________________
%                      A rotated Two column Table in landscape
%-------------------------------------------------------------
\begin{sidewaystable*}
\caption{Summary for ISOCAM sources with mid-IR excess
(YSO candidates).}\label{YSOtable}
\centering
\begin{tabular}{crrlcl}
\hline\hline
ISO-L1551 & $F_{6.7}$~[mJy] & $\alpha_{6.7-14.3}$
& YSO type$^{d}$ & Status & Comments\\
\hline
  \multicolumn{6}{c}{\it New YSO candidates}\\ % To combine 6 columns into a single one
\hline
  1 & 1.56 $\pm$ 0.47 & --    & Class II$^{c}$ & New & Mid\\
  2 & 0.79:           & 0.97: & Class II ?     & New & \\
  3 & 4.95 $\pm$ 0.68 & 3.18  & Class II / III & New & \\
  5 & 1.44 $\pm$ 0.33 & 1.88  & Class II       & New & \\
\hline
  \multicolumn{6}{c}{\it Previously known YSOs} \\
\hline
  61 & 0.89 $\pm$ 0.58 & 1.77 & Class I & \object{HH 30} & Circumstellar disk\\
  96 & 38.34 $\pm$ 0.71 & 37.5& Class II& MHO 5          & Spectral type\\
\hline
\end{tabular}
\end{sidewaystable*}
%_____________________________________________________________
%                      A rotated One column Table in landscape
%-------------------------------------------------------------
\begin{sidewaystable}
\caption{Summary for ISOCAM sources with mid-IR excess
(YSO candidates).}\label{YSOtable}
\centering
\begin{tabular}{crrlcl}
\hline\hline
ISO-L1551 & $F_{6.7}$~[mJy] & $\alpha_{6.7-14.3}$
& YSO type$^{d}$ & Status & Comments\\
\hline
  \multicolumn{6}{c}{\it New YSO candidates}\\ % To combine 6 columns into a single one
\hline
  1 & 1.56 $\pm$ 0.47 & --    & Class II$^{c}$ & New & Mid\\
  2 & 0.79:           & 0.97: & Class II ?     & New & \\
  3 & 4.95 $\pm$ 0.68 & 3.18  & Class II / III & New & \\
  5 & 1.44 $\pm$ 0.33 & 1.88  & Class II       & New & \\
\hline
  \multicolumn{6}{c}{\it Previously known YSOs} \\
\hline
  61 & 0.89 $\pm$ 0.58 & 1.77 & Class I & \object{HH 30} & Circumstellar disk\\
  96 & 38.34 $\pm$ 0.71 & 37.5& Class II& MHO 5          & Spectral type\\
\hline
\end{tabular}
\end{sidewaystable}
%
%_____________________________________________________________
%                              Table longer than a single page
%-------------------------------------------------------------
% All long tables will be placed automatically at the end of the document
%
\longtab{
\begin{longtable}{lllrrr}
\caption{\label{kstars} Sample stars with absolute magnitude}\\
\hline\hline
Catalogue& $M_{V}$ & Spectral & Distance & Mode & Count Rate \\
\hline
\endfirsthead
\caption{continued.}\\
\hline\hline
Catalogue& $M_{V}$ & Spectral & Distance & Mode & Count Rate \\
\hline
\endhead
\hline
\endfoot
%%
Gl 33    & 6.37 & K2 V & 7.46 & S & 0.043170\\
Gl 66AB  & 6.26 & K2 V & 8.15 & S & 0.260478\\
Gl 68    & 5.87 & K1 V & 7.47 & P & 0.026610\\
         &      &      &      & H & 0.008686\\
Gl 86
\footnote{Source not included in the HRI catalog. See Sect.~5.4.2 for details.}
         & 5.92 & K0 V & 10.91& S & 0.058230\\
\end{longtable}
}
%
%_____________________________________________________________
%                              Table longer than a single page
%                                             and in landscape
%  In the preamble, use:       \usepackage{lscape}
%-------------------------------------------------------------
% All long tables will be placed automatically at the end of the document
%
\longtab{
\begin{landscape}
\begin{longtable}{lllrrr}
\caption{\label{kstars} Sample stars with absolute magnitude}\\
\hline\hline
Catalogue& $M_{V}$ & Spectral & Distance & Mode & Count Rate \\
\hline
\endfirsthead
\caption{continued.}\\
\hline\hline
Catalogue& $M_{V}$ & Spectral & Distance & Mode & Count Rate \\
\hline
\endhead
\hline
\endfoot
%%
Gl 33    & 6.37 & K2 V & 7.46 & S & 0.043170\\
Gl 66AB  & 6.26 & K2 V & 8.15 & S & 0.260478\\
Gl 68    & 5.87 & K1 V & 7.47 & P & 0.026610\\
         &      &      &      & H & 0.008686\\
Gl 86
\footnote{Source not included in the HRI catalog. See Sect.~5.4.2 for details.}
         & 5.92 & K0 V & 10.91& S & 0.058230\\
\end{longtable}
\end{landscape}
}
%
%
%__________________________________________________________________
%        Appendices have to be placed at the end, after
%                                        \end{thebibliography}
%
%        After compilation, Appendices will be placed automatically
%        on a new page.
%------------------------------------------------------------------
\end{thebibliography}

\begin{appendix} %First  appendix
\section{Background galaxy number counts and shear noise-levels}
Because the optical images used in this analysis...
\begin{figure*}%f1
\includegraphics[width=10.9cm]{1787f23.eps}
\caption{Shown in greyscale is a...}
\label{cl12301}}
\end{figure*}

In this case....
\begin{figure*}
\centering
\includegraphics[width=16.4cm,clip]{1787f24.ps}
\caption{Plotted above...}
\label{appfig}
\end{figure*}

Because the optical images...

\section{Title of Second appendix.....} %Second appendix
These studies, however, have faced...
\begin{table}
\caption{Complexes characterisation.}\label{starbursts}
\centering
\begin{tabular}{lccc}
\hline \hline
Complex & $F_{60}$ & 8.6 &  No. of  \\
...
\hline
\end{tabular}
\end{table}

The second method produces...
\end{appendix}
%
%
\end{document}

%
%-------------------------------------------------------------
% For the appendices, table longer than a single page
%                 In the preamble for landscape case, use :
%                                          \usepackage{lscape}
%-------------------------------------------------------------
\documentclass{aa}
\usepackage[varg]{txfonts}
\usepackage{graphicx}
\usepackage{lscape}

\begin{document}
text of the paper
% Table will be placed automatically at the end of the document, after the whole appendices.
%For example table 3
\longtab[3]{
\begin{longtable}{lrcrrrrrrrrl}
\caption{Line data and abundances ...}\\
\hline
\hline
Def & mol & Ion & $\lambda$ & $\chi$ & $\log gf$ & N & e &  rad & $\delta$ & $\delta$
red & References \\
\hline
\endfirsthead
\caption{continued.} \\
\hline
Def & mol & Ion & $\lambda$ & $\chi$ & $\log gf$ & B & C &  rad & $\delta$ & $\delta$
red & References \\
\hline
\endhead
\hline
\endfoot
\hline
\endlastfoot
A & CH & 1 &3638 & 0.002 & $-$2.551 &  &  &  & $-$150 & 150 &  Jorgensen et al. (1996) \\
\end{longtable}
}% End longtab

% Or for landscape, large table:

%For example table 1
\longtab[1]{
\begin{landscape}
\begin{longtable}{lrcrrrrrrrrl}
...
\end{longtable}
\end{landscape}
}% End longtab



%%%%% End of aa.dem
