%     aa.dem
% AA vers. 9.1, LaTeX class for Astronomy & Astrophysics
% demonstration file
%                                                       (c) EDP Sciences
%-----------------------------------------------------------------------
%
%\documentclass[referee]{aa} % for a referee version
%\documentclass[onecolumn]{aa} % for a paper on 1 column
%\documentclass[longauth]{aa} % for the long lists of affiliations
%\documentclass[letter]{aa} % for the letters
%\documentclass[bibyear]{aa} % if the references are not structured
%                              according to the author-year natbib style

%
\documentclass[twocolumn]{aa}

%
\usepackage{graphicx}
%%%%%%%%%%%%%%%%%%%%%%%%%%%%%%%%%%%%%%%%
\usepackage{txfonts}
%%%%%%%%%%%%%%%%%%%%%%%%%%%%%%%%%%%%%%%%
%\usepackage[options]{hyperref}
% To add links in your PDF file, use the package "hyperref"
% with options according to your LaTeX or PDFLaTeX drivers.
%
\hbadness=99999
\begin{document}


   \title{Modelling the formation of the GD-1 stellar stream inside a host with a fermionic dark matter core-halo distribution.}

   % \subtitle{I. Overviewing the $\kappa$-mechanism}

   \author{Martín F. Mestre\inst{1,2}\fnmsep\thanks{
         \email{mmestre@fcaglp.unlp.edu.ar}
      }
      \and
      Carlos R. Argüelles\inst{1,2}
      \and
      Daniel D. Carpintero\inst{1,2}
   }

   \institute{Instituto de Astrof{\'i}sica de La Plata (CONICET-UNLP), Paseo del
               Bosque S/N, La Plata (1900), Buenos Aires, Argentina\\
               \and
              Facultad de Ciencias Astron{\'o}micas y Geof{\'i}sicas de La Plata (UNLP),
              Paseo del Bosque S/N, La Plata (1900), Buenos Aires, Argentina\\
             }

   \date{Received February 29, 2023; accepted February 30, 2023}

% \abstract{}{}{}{}{}
% 5 {} token are mandatory

  \abstract
  % context heading (optional)
  % {} leave it empty if necessary
   {Stellar streams are a consequence of the tidal forces produced by a host galaxy on its satellites (i.e. globular clusters and dwarf spheroidals).
   As the self-gravity of stellar streams is almost negligible, they constitute excellent probes of the gravitational potential of the host galaxy. For this reason, some Milky Way stellar streams have been used to put constraints on the dark matter (DM) total mass and shape, under empirical DM distributions (i.e. NFW, logarithmic, etc). Within the set of non-empirical DM distributions, there exists a DM model
   deduced from first principles by means of the maximization of the coarse-grained entropy for self-gravitating fermions, the MEPP, after Maximum Entropy Production Principle. MEPP profiles have a rich
   variety of behaviours, being a large subfamily characterized by a dense core of mass
   $\approx4.1\times10^6 M_\odot$ (that can mimick a black hole regarding the orbits of the S-stars at Sagittario A*) and an extended halo with the plasticity of a Burkert profile.}
  % aims heading (mandatory)
   {In this work we attempt to model the GD-1 stellar stream using a spherical MEPP distribution which, at the same time, is in sintony with previous fits to the S-stars.}
  % methods heading (mandatory)
   {For that purpose we used a genetic algorithm in order to fit both the stream orbit's initial conditions and the fermionic halo. We modelled the barionic potential with a bulge and two disks (thin and thick) with fixed parameters according to the recent literature. The stream observable is 6D phase-space data from the Gaia DR2 survey.}
  % results heading (mandatory)
   {We were able to find good fits for a 1D continuous subfamily of models parametrized by the fermion
   mass going from $56$~keV/$c^2$ to $~360$~keV/$c^2$, respectively corresponding to core radii going from
   $10^3$ to 10 Schwarzschild radii. For smaller and larger values of the fermion mass, there is no solution that simultaneously fits the GD-1 stream and the S-stars. The solutions have a virial radius of 28 kpc and a virial mass of $1.4\times10^{11} M_\odot$, the latter being at $2\sigma$ from previous Milky Way DM halo mass estimates using the Sagittarius stream. We do not assume the velocity of the local standard of rest ($v_\mathrm{lsr}$) and the result gives $v_\mathrm{lsr}\approx244$ km/s, in agreement with recent independent estimates.}
  % conclusions heading (optional), leave it empty if necessary
   {}

   \keywords{Galaxy: kinematics and dynamics --
             Galaxy: stellar streams --
             dark matter
               }
   \titlerunning{The GD-1 stream inside a fermionic dark matter halo.}

   \maketitle
%________________________________________________________________

\section{Introduction}

Stellar streams probe the acceleration field produced by the Milky Way (MW).
This information together with barionic measurements help in making claims about the
dark matter halo.

\section{Methodology}
In this section we explain the observables and methods used in this research.

\subsection{Observables and assumed measurements}
\label{observables}

The main observables used in this project are materialized by the polynomial fits performed by
\citet{Ibata_2020} on the GD-1\\ stream using astrometry (Gaia DR2), photometry and high-precision spectroscopy datasets together with the analysis of the {\sc streamfinder} algorithm.
Those polinomials are the following:
\begin{eqnarray}
   \label{Ibata_polyn}
   \phi_2  &=& 0.008367\phi_1^3-0.05332\phi_1^2-0.07739\phi_1-0.02007 \\
   D &=& -4.302\phi_1^5-11.54\phi_1^4-7.161\phi_1^3 +5.985\phi_1^2 \nonumber\\
      &&+ 8.595\phi_1+10.36\\
   v_h &=&  90.68\phi_1^3+204.5\phi_1^2-254.2\phi_1-261.5\\
   \tilde{\mu}_\alpha &=& 3.794\phi_1^3+9.467\phi_1^2+1.615\phi_1-7.844\\
   \mu_\delta &=& -1.225\phi_1^3+8.313\phi_1^2+18.68\phi_1-3.95,
   \label{Ibata_polynb}
\end{eqnarray}
with $\phi_1$ and $\phi_2$ in radians, $D$ in kpc and, $\tilde{\mu}_\alpha=\mu_\alpha cos \delta$ and $\mu_\delta$ in mas/yr. These quantities correspond respectively to longitude and latitude in GD-1 celestial frame of reference~\citep{Koposov_2010}, heliocentric distance, proper motion in right ascension and declination. The domain of this polinomials is limited to $-90<\phi_1[^\circ]<10$.
To have our observable data points we sample the domain with 100 points ($\phi_1^{(i)}$ for $i=1,...,100$)
and evaluate the polinomials there in order to have a complete set of observables ($\phi_1^{(i)}$, $\phi_2^{(i)}$, $D^{(i)}$, $\tilde{\mu}_\alpha^{(i)}$, $\mu_\delta^{(i)}$).
In one of the experiments we will consider an observable of a different nature
and related to the density concentration at the origin, of size $\lesssim 1$ milliparsec,
characteristic of the fermionic halo distribution. By convention, the core radius is defined
as the radius when the circular velocity reaches its first maximum.
The constraint for the mass of the core of the distribution is assumed to be $m_{\rm{core}}=3.5\times10^6 M_\odot$ in agreement with~\citet{2020A&A...641A..34B,2021MNRAS.505L..64B,2022MNRAS.511L..35A}.

The assumed measurements used in this paper are the
galactocentric distance of the sun $R_\odot=8.122$ kpc~\citep{2018A&A...615L..15G}, the sun's peculiar
velocity $\boldsymbol{v}_{\odot p} = (11.1, 12.24, 7.25)$ km/s~\citep{Shonrich} and the value of the universal constant
of gravitation $G= 4.3009\times10^{-6}$ kpc (km/s)$^2$ $M_\odot^{-1}$.


\subsection{The fermionic dark matter model}
Our dark matter model consists of a spherical and isotropic distribution of fermions at finite temperature in hydrostatic equilibrium subject to the law of General Relativity (i.e. T.O.V equations) as defined in~\cite{arguelles_novel_2018} with the particularity that we have made a few change of variables in order to make the numerical computation more robust. The system of equations thus obtained is as follows:

\begin{eqnarray}
   \label{sode}
   \frac{d\nu}{d\zeta} & = & \frac{1}{2}\left[e^{z}+e^{2\zeta}\frac{P(r)}{\rho_{rel}c^2}\right][1-e^{z}]^{-1},\\
   \frac{dz}{d\zeta} & = &-1+e^{(2\zeta-z)}\frac{\rho(r)}{\rho_{rel}},
\end{eqnarray}
where
\begin{eqnarray}
   \zeta &=& \ln(r/R),\\
   z &=&\ln\psi,\quad \psi = \frac{M(r)}{M}\frac{R}{r},\label{mass_dm}\\
   \rho(r)&=&\frac{4\rho_{\rm rel}}{\sqrt{\pi}}\int^\infty_1\epsilon^2[\epsilon^2-1]^{1/2}f(r,\epsilon)d\epsilon,\\
   P(r)&=&\frac{4\rho_{\rm rel}c^2}{3\sqrt{\pi}}\int^\infty_1[\epsilon^2-1]^{3/2}f(r,\epsilon)d\epsilon,\\
   R^2 &=& \frac{c^2}{8\pi G \rho_{\rm{rel}}},\\
   M &=& 4\pi R^3 \rho_{\rm{rel}},\\
   \rho_{\rm{rel}}&=& \frac{ G\pi^{3/2} m^4 c^3}{h^3},
\end{eqnarray}
where $c$ is the speed of light, $h$ is the Planck constant, $m$ is the fermion mass, $M(r)$ is the mass enclosed up to radius $r$, $\rho(r)$ is the density, $P(r)$ is the pressure and $\nu(r)$ is the metric exponent, i.e. we use the convention $g_{00}=e^{\nu(r)}$. The phase-space function ($f$) is given by a Fermi-Dirac distribution with energy cutoff:
\begin{equation}
f(r,\epsilon)=
   \begin{cases}
      \frac{1-e^{[\epsilon-\epsilon_c(r)]/\beta(r)}}
      {1+e^{[\epsilon-\alpha(r)]/\beta(r)}}\quad \epsilon \leq \epsilon_c(r)\\
      0\quad \epsilon > \epsilon_c(r),
   \end{cases}
\end{equation}
where $\alpha(r)$ is the chemical potential (including rest mass), $\epsilon(r)$ is the cutoff
energy (including rest mass) and $\beta(r)$ is the temperature variable.

The above equations should be complemented with two thermodinamic equilibrium conditions given
in~\citet{PhysRev.35.904} and~\citet{RevModPhys.21.531} together with the condition of energy conservation
along the geodesic given in~\citet{1989A&A...221....4M}:
\begin{equation}
   \frac{1}{\alpha}\frac{d\alpha}{dr}=\frac{1}{\beta}\frac{d\beta}{dr}=
   \frac{1}{\epsilon_c}\frac{d\epsilon_c}{dr}=-\frac{1}{2}\frac{d\nu}{dr}.
   \label{geod_energy}
\end{equation}

It is not possible to integrate this equations from $r=0$ because the right hand the change of variables $\zeta(r)$ is divergent at the origin. Nevertheless, it is possible
to reach the following approximations for the initial conditions at a value $r_{\rm{min}}\gtrsim 0$:
\begin{eqnarray}
   \nu(r_{\rm{min}}) &\approx& \frac{1}{6}\frac{\rho_0}{\rho_{\rm{rel}}}\left[\frac{r_{\rm{min}}}{R}\right]^2\equiv\tau, \\
   \psi(r_{\rm{min}})&\approx&\frac{1}{3}\frac{\rho_0}{\rho_{\rm{rel}}}\left[\frac{r_{\rm{min}}}{R}\right]^2=2\tau,
\end{eqnarray}
which implies
\begin{equation}
   \frac{r_{\rm{min}}}{R}=\sqrt{6\tau\frac{\rho_{\rm{rel}}}{\rho_0}},
\end{equation}
where $\rho_0\equiv \rho(0)$.
It can be seen that the right hand side of Eq.~\ref{sode} does not depend on the metric so we can add
a constant $\nu_0$ to the solution in order to satisfy a condition of continuity with the Schwarzschild metric at the border of the fermion distribution, obtaining:
$$\nu_0 = 2\ln\left(\frac{\beta_b}{\beta_0}\sqrt{1-\psi_b}\right),$$
where $\psi_b$ and $\beta_b$ are quantities evaluated at the border.

The system of equations thus obtained has four free parameters: $m$, $\alpha_0=\alpha(0)$,
$\beta_0=\beta(0)$, and $\epsilon_{c0}=\epsilon_c(0)$ but in practice we will use the following related
quantities as parameters:
$m$, $\beta_0$, $\theta_0=(\alpha_0-1)/\beta_0$, and $W_0=(\epsilon_{c0}-1)/\beta_0$.

Equations~\ref{sode}-\ref{geod_energy} are solved using
a {\scshape{Python}}~\citep{van1995python} script\footnote{
\url{https://github.com/martinmestre/stream-fit/blob/main/likelihood_grid/optim/model_def.py}
}
that makes use of the {\scshape{NumPy}}~\citep{2020SciPy-NMeth} and {\scshape{SciPy}}~\citep{harris2020array} packages. We tried all the solvers available in the latter package but the only one that
could properly integrate these equations was the \texttt{LSODA} algorithm. We used relative and absolute tolerance parameters respectively given by \texttt{rtol}$=5\times10^{-14}$ and \texttt{atol}$=0$.


\subsection{Milky Way and stream models}

We model our Galaxy with a combination of the fermionic dark halo recently described, whose parameters will be determined in this work, plus a barionic component fixed and identical to the one in Model I of~\citet{2017A&A...598A..66P}. Here we name this full Galaxy model as Fermionic-MW.

Appart from our model, for some intermediary tasks, we will make use of the Galactic model fitted
by~\citet{2019MNRAS.486.2995M} which consists in the {\texttt{MWPotential2014}}
model with an axisymmetric NFW profile. This fitted model corresponds to a circular velocity at the position of the sun of
$v_c(R_\odot)=244 \pm 4 \rm{ km s}^{-1}$ and a $z-$flattening of the dark matter density distribution of $q_\rho=0.82^{+0.25}_{-0.13}$. Here we name this Galaxy model as NFW-MW.

As GD-1 is a dynamically cold stream, with its stars keeping a large degree of correlation, its almost
one-dimensional distribution in phase-space can be well approximated with the orbit of its progenitor.
Another facts in this direction, are that there has not been any observation of the tidal arms feature
nor of the progenitor's position~\citep{2019MNRAS.486.2995M,10.1093/mnras/sty677,10.1093/mnras/sty1338,Price-Whelan_2018}.

In the code, the orbit model is computed starting from initial conditions in spherical equatorial coordinates:
$\alpha$, $\delta$, $D_h$, $\mu_\alpha \cos\delta$, $\mu_\delta$, $v_h$, which are respectively, RA, declination, heliocentric distance, proper motion in $\alpha$ times the cosine of $\delta$, proper motion in $\delta$ and helicentric radial velocity.
The code uses the {\scshape{Astropy}} ecosystem~\citep{astropy:2022, astropy:2018, astropy:2013} in order to transform the initial condition to galactocentric coordinates assuming a Galactocentric reference frame with the sun at the position $\boldsymbol{x}_\odot=(-R_\odot,0,0)$ and a sun's velocity given by the sum of the circular velocity at the position of the sun and
the sun's peculiar velocity: $\boldsymbol{v}_\odot=(11.1, v_c(R_\odot)+12.24, 7.25)$. The circular velocity is dependent on the model and the position and given by

\begin{eqnarray}
   v_c^2(R_\odot)&=& R_\odot||\nabla \Phi(\boldsymbol{x})||_{\boldsymbol{x}=\boldsymbol{x}_\odot}\nonumber\\
   &=& R_\odot||\nabla \Phi_{\rm{B}}(\boldsymbol{x})||_{\boldsymbol{x}=\boldsymbol{x}_\odot}+
   G \frac{M_{\rm{DM}}(R_\odot)}{R_\odot},
\end{eqnarray}
where $\Phi$ is the total potential, $\Phi_B$ is the potential generated by the three barionic components and $M(r)$ is the enclosed dark matter mass as defined in Eq.~(\ref{mass_dm}). We have used the spherical symmetry property of the DM distribution in order to relate the acceleration with the enclosed mass.

A second step is to integrate the orbit forwards and backwards in time during a time interval of $\Delta t=0.2$ Gyr, starting in both cases from a given initial condition for the progenitor. In the next sections
we will explain how this initial condition was chosen for some simulations and fitted for others. The integrator used is a Runge-Kutta of order eight (\texttt{DOP853} called from {\scshape{SciPy}}'s \texttt{solve\_ivp} function) with relative and absolute tolerance parameters respectively given by
\texttt{rtol}$=5\times10^{-14}$ and \texttt{atol}$=0.5\times10^{-14}$.

After having the stream's orbit integrated, we transform the orbit to both the ICRS and the GD-1 frame of coordinates. For the latter we used the \texttt{GD1Koposov10} class defined in the {\scshape{Gala}} package~\citep{gala,adrian_price_whelan_2020_4159870} which uses the transformation matrix defined
by~\citet{Koposov_2010}.
After these two transformation we obtain the orbit expressed in the observable variables ($\phi_1$, $\phi_2$, $D$, $v_h$, $\tilde{\mu}_\alpha$, $\mu_\delta$) used in Eqs.~(\ref{Ibata_polyn})-(\ref{Ibata_polynb}).
Finally, we build interpolators of these variables as a function of $\phi_1$ that will be used, together with the observed data defined in Sec.~\ref{observables}, to evaluate the following $\chi^2$ stream function:
\begin{eqnarray}
   \label{chi2}
   \chi^2_{\rm{stream}} &=& \chi^2_{\phi_2} + \chi^2_{D}+ \chi^2_{v_h}+\chi^2_{\tilde{\mu}_\alpha}+ \chi^2_{\mu_\delta},\\
   \chi^2_{\eta} &=& \frac{1}{\sigma_\eta^2}\sum_{i=1}^{100} \big(\eta^{(i)}-\eta(\phi_1^{(i)})\big)^2,
\end{eqnarray}
where
$\eta$ represents each of the four corresponding independent variables ($\phi_2$, $D$, $v_h$, $\tilde{\mu}_\alpha$, $\mu_\delta$) and $\sigma_\eta$ is the average dispersion of the stream data points taken by
ocular inspection from Figs.~1,3,4 of~\citet{Ibata_2020}:
$\sigma_{\phi_2}=0.5^\circ$,
$\sigma_{D}= 1.5~\rm{kpc}$,
$\sigma_{v_h} = 10~\rm{km/s}$ and
$\sigma_{\tilde{\mu}_\alpha}= \sigma_{\delta}= 2~\rm{mas/yr}$.
Thus $\chi^2_{\rm{stream}}$ measures the departure of the model from the stream observations.
For some fits we will also consider the departure of the model from a mass constraint in the core
of the distribution:
\begin{equation}
   \chi^2_{\rm{core}}=\frac{(m-m_{\rm{core}})^2}{\sigma_m^2},
\end{equation}
where $m_{\rm{core}}$ was defined in Sec.~\ref{observables} and
$\sigma_m$ was tested between 10\% and 50\% of its value, so sometimes we will use the following compound function:
\begin{equation}
   \label{chi2_full}
   \chi^2_{\rm{full}}=\chi^2_{\rm{stream}}+\chi^2_{\rm{core}}.
\end{equation}


\subsection{Optimization algorithms}
Our objective consists in fitting our MW model by minimizing the $\chi^2$ function given by Eqs.~(\ref{chi2},\ref{chi2_full}).  In order to do so we use two optimization algorithms in this work.
One of them is {\scshape{SciPy}}'s
\texttt{optimize.differential\_evolution} algorithm. This genetic architecture
was used with metaparemeters given by
\texttt{strategy}=\texttt{"best2bin"}, \texttt{maxiter}=100, \texttt{popsize}=50,
\texttt{tol}=$5\times10^{-8}$ and \texttt{atol}=0. This method can be run in parallel with shared memory,
so a convergence test was made in a cluster's node with 24 processors, varying the values of the metaparameters (e.g. doubling the values of \texttt{maxiter} and \texttt{popsize}, setting \texttt{strategy}=\texttt{"best1bin"} and combinations).

The other algorithm is {\scshape{NLopt}}'s
\texttt{LN\_NELDERMEAD} called from a  {\scshape{Julia}}~\citep{bezanson2017julia}
environment. We used default values for the all its metaparameters except for
\texttt{reltol}=$5\times10^{-5}$.

These two optimizators where used for different purposes that will be explained
in the following section along the results.


\section{Results}

\section{Discussion}

\section{Conclusions}


%______________________________________________________________


\begin{acknowledgements}
    Thank to people, institutions and codes used.
\end{acknowledgements}

% WARNING
%-------------------------------------------------------------------
% Please note that we have included the references to the file aa.dem in
% order to compile it, but we ask you to:
%
% - use BibTeX with the regular commands:
%   \bibliographystyle{aa} % style aa.bst
%   \bibliography{Yourfile} % your references Yourfile.bib
%
% - join the .bib files when you upload your source files
%-------------------------------------------------------------------
\bibliographystyle{aa} % style aa.bst
\bibliography{refs} % your references Yourfile.bib
%

\end{document}


