%     aa.dem
%DIF LATEXDIFF DIFFERENCE FILE
%DIF DEL 48626corr.tex   Thu May 30 12:32:48 2024
%DIF ADD approved.tex    Thu May 30 17:07:26 2024
% AA vers. 9.1, LaTeX class for Astronomy & Astrophysics
% demonstration file
%                                                       (c) EDP Sciences
%-----------------------------------------------------------------------
%DIF 6-7c6-7
%DIF < %
%DIF < \documentclass[referee]{aa} % for a referee version
%DIF -------
% % %DIF >
% \documentclass[referee]{aa} % for a referee version %DIF >
%DIF -------
%\documentclass[onecolumn]{aa} % for a paper on 1 column
%\documentclass[longauth]{aa} % for the long lists of affiliations
%\documentclass[letter]{aa} % for the letters
%\documentclass[bibyear]{aa} % if the references are not structured
%                              according to the author-year natbib style

%DIF 14c14
%DIF < %\documentclass[twocolumn]{aa}
%DIF -------
\documentclass[twocolumn]{aa} %DIF >
%DIF -------

\usepackage{graphicx}
\usepackage{txfonts}
\usepackage[dvipsnames]{xcolor}
\usepackage{soul}
\usepackage{hyperref}
\usepackage[capitalise]{cleveref}
\usepackage{siunitx}
%DIF 23c23
%DIF < %\usepackage{orcidlink}
%DIF -------
\usepackage{orcidlink} %DIF >
%DIF -------

% ------------------------------------------------------------------
% sets
% ------------------------------------------------------------------

\def \L {\mathbb{L}}
\def \N {\mathbb{N}} % natural numbers
\def \Z {\mathbb{Z}} % integer numbers
\def \Q {\mathbb{Q}} % rational numbers
\def \R {\mathbb{R}} % real numbers
\def \C {\mathbb{C}} % complex numbers

% ------------------------------------------------------------------
% common symbols
% ------------------------------------------------------------------

\def \e         {\mathrm{e}} % euler number
\def \d         {\mathrm{d}} % differential
\def \i         {\mathfrak{i}} % imaginary unit
\def \const     {\mathrm{const}} % constant expression
\def \other     {\mathrm{else}} % else expression
\def \um        {\text{-}} % small minus sign (useful for exponential or index notation)

% ------------------------------------------------------------------
% operators and functions
% ------------------------------------------------------------------

\def \var		{\mathrm{\delta}}
\DeclareMathOperator{\sgn}{sgn}
\DeclareMathOperator{\diag}{diag}

% ------------------------------------------------------------------
% fractions
% ------------------------------------------------------------------

\newcommand\FRAC[2]{\frac{\displaystyle #1}{\displaystyle #2}} % large frac

% ------------------------------------------------------------------
% brackets
% ------------------------------------------------------------------

\newcommand \brackets[2][]{\left[\vphantom{#1}#2\right]} % [...]
\def \parantheses  #1{\left(#1\right)} % (...)
\def \braces       #1{\left\{#1\right\}} % {...}
\def \angles       #1{\left\langle#1\right\rangle} % <...>
\def \pipes        #1{\left|#1\right|} % |...|
\def \doublepipes  #1{\left\|#1\right\|} % ||...||

% ------------------------------------------------------------------
% derivatives
% ------------------------------------------------------------------

\def \diff				#1#2{\frac{\d#1}{\d#2}}
\def \ddiff				#1#2{\frac{\d^2#1}{\d#2^2}}
\def \ndiff				#1#2#3{\frac{\d^{#3}#1}{\d#2^{#3}}}

\def \diffat			#1#2#3{\at{\diff{#1}{#2}}{#3}}
\def \ddiffat			#1#2#3{\at{\ddiff{#1}{#2}}{#3}}
\def \ndiffat			#1#2#3#4{\at{\ndiff{#1}{#2}{#3}}{#4}}

\def \diffpart		#1#2{\frac{\partial #1}{\partial #2}}
\def \ddiffpart		#1#2{\frac{\partial^2 #1}{\partial #2^2}}
\def \ndiffpart		#1#2#3{\frac{\partial^{#3}#1}{\partial#2^{#3}}}

\def \diffpartat	#1#2#3{\at{\diffpart{#1}{#2}}{#3}}
\def \ddiffpartat	#1#2#3{\at{\ddiffpart{#1}{#2}}{#3}}
\def \ndiffpartat	#1#2#3#4{\at{\ndiffpart{#1}{#2}{#3}}{#4}}

\def \todiffpart	#1#2{\frac{\partial #1}{\partial #2}\d#2}

% make tables less compact
\renewcommand{\arraystretch}{1.2}

% colors
\newcommand{\rojo}[1]{{\color{red} #1}}
\newcommand{\verde}[1]{{\color{ForestGreen} #1}}

\hypersetup{
    colorlinks,
    linkcolor={red!50!black},
    citecolor={blue!50!black},
    urlcolor={blue!80!black}
}

% define SI units
\DeclareSIUnit \year {yr}
\DeclareSIUnit \parsec {pc}
\DeclareSIUnit \eV {eV}
\DeclareSIUnit \Msun {M_\odot}

% change SI range
\sisetup{
	range-phrase	= \text{ -- },
	range-units		= single
}


\hbadness=99999
%DIF PREAMBLE EXTENSION ADDED BY LATEXDIFF
%DIF UNDERLINE PREAMBLE %DIF PREAMBLE
\RequirePackage[normalem]{ulem} %DIF PREAMBLE
\RequirePackage{color}\definecolor{RED}{rgb}{1,0,0}\definecolor{BLUE}{rgb}{0,0,1} %DIF PREAMBLE
\providecommand{\DIFaddtex}[1]{{\protect\color{blue}\uwave{#1}}} %DIF PREAMBLE
\providecommand{\DIFdeltex}[1]{{\protect\color{red}\sout{#1}}}                      %DIF PREAMBLE
%DIF SAFE PREAMBLE %DIF PREAMBLE
\providecommand{\DIFaddbegin}{} %DIF PREAMBLE
\providecommand{\DIFaddend}{} %DIF PREAMBLE
\providecommand{\DIFdelbegin}{} %DIF PREAMBLE
\providecommand{\DIFdelend}{} %DIF PREAMBLE
\providecommand{\DIFmodbegin}{} %DIF PREAMBLE
\providecommand{\DIFmodend}{} %DIF PREAMBLE
%DIF FLOATSAFE PREAMBLE %DIF PREAMBLE
\providecommand{\DIFaddFL}[1]{\DIFadd{#1}} %DIF PREAMBLE
\providecommand{\DIFdelFL}[1]{\DIFdel{#1}} %DIF PREAMBLE
\providecommand{\DIFaddbeginFL}{} %DIF PREAMBLE
\providecommand{\DIFaddendFL}{} %DIF PREAMBLE
\providecommand{\DIFdelbeginFL}{} %DIF PREAMBLE
\providecommand{\DIFdelendFL}{} %DIF PREAMBLE
%DIF HYPERREF PREAMBLE %DIF PREAMBLE
\providecommand{\DIFadd}[1]{\texorpdfstring{\DIFaddtex{#1}}{#1}} %DIF PREAMBLE
\providecommand{\DIFdel}[1]{\texorpdfstring{\DIFdeltex{#1}}{}} %DIF PREAMBLE
\newcommand{\DIFscaledelfig}{0.5}
%DIF HIGHLIGHTGRAPHICS PREAMBLE %DIF PREAMBLE
\RequirePackage{settobox} %DIF PREAMBLE
\RequirePackage{letltxmacro} %DIF PREAMBLE
\newsavebox{\DIFdelgraphicsbox} %DIF PREAMBLE
\newlength{\DIFdelgraphicswidth} %DIF PREAMBLE
\newlength{\DIFdelgraphicsheight} %DIF PREAMBLE
% store original definition of \includegraphics %DIF PREAMBLE
\LetLtxMacro{\DIFOincludegraphics}{\includegraphics} %DIF PREAMBLE
\newcommand{\DIFaddincludegraphics}[2][]{{\color{blue}\fbox{\DIFOincludegraphics[#1]{#2}}}} %DIF PREAMBLE
\newcommand{\DIFdelincludegraphics}[2][]{% %DIF PREAMBLE
\sbox{\DIFdelgraphicsbox}{\DIFOincludegraphics[#1]{#2}}% %DIF PREAMBLE
\settoboxwidth{\DIFdelgraphicswidth}{\DIFdelgraphicsbox} %DIF PREAMBLE
\settoboxtotalheight{\DIFdelgraphicsheight}{\DIFdelgraphicsbox} %DIF PREAMBLE
\scalebox{\DIFscaledelfig}{% %DIF PREAMBLE
\parbox[b]{\DIFdelgraphicswidth}{\usebox{\DIFdelgraphicsbox}\\[-\baselineskip] \rule{\DIFdelgraphicswidth}{0em}}\llap{\resizebox{\DIFdelgraphicswidth}{\DIFdelgraphicsheight}{% %DIF PREAMBLE
\setlength{\unitlength}{\DIFdelgraphicswidth}% %DIF PREAMBLE
\begin{picture}(1,1)% %DIF PREAMBLE
\thicklines\linethickness{2pt} %DIF PREAMBLE
{\color[rgb]{1,0,0}\put(0,0){\framebox(1,1){}}}% %DIF PREAMBLE
{\color[rgb]{1,0,0}\put(0,0){\line( 1,1){1}}}% %DIF PREAMBLE
{\color[rgb]{1,0,0}\put(0,1){\line(1,-1){1}}}% %DIF PREAMBLE
\end{picture}% %DIF PREAMBLE
}\hspace*{3pt}}} %DIF PREAMBLE
} %DIF PREAMBLE
\LetLtxMacro{\DIFOaddbegin}{\DIFaddbegin} %DIF PREAMBLE
\LetLtxMacro{\DIFOaddend}{\DIFaddend} %DIF PREAMBLE
\LetLtxMacro{\DIFOdelbegin}{\DIFdelbegin} %DIF PREAMBLE
\LetLtxMacro{\DIFOdelend}{\DIFdelend} %DIF PREAMBLE
\DeclareRobustCommand{\DIFaddbegin}{\DIFOaddbegin \let\includegraphics\DIFaddincludegraphics} %DIF PREAMBLE
\DeclareRobustCommand{\DIFaddend}{\DIFOaddend \let\includegraphics\DIFOincludegraphics} %DIF PREAMBLE
\DeclareRobustCommand{\DIFdelbegin}{\DIFOdelbegin \let\includegraphics\DIFdelincludegraphics} %DIF PREAMBLE
\DeclareRobustCommand{\DIFdelend}{\DIFOaddend \let\includegraphics\DIFOincludegraphics} %DIF PREAMBLE
\LetLtxMacro{\DIFOaddbeginFL}{\DIFaddbeginFL} %DIF PREAMBLE
\LetLtxMacro{\DIFOaddendFL}{\DIFaddendFL} %DIF PREAMBLE
\LetLtxMacro{\DIFOdelbeginFL}{\DIFdelbeginFL} %DIF PREAMBLE
\LetLtxMacro{\DIFOdelendFL}{\DIFdelendFL} %DIF PREAMBLE
\DeclareRobustCommand{\DIFaddbeginFL}{\DIFOaddbeginFL \let\includegraphics\DIFaddincludegraphics} %DIF PREAMBLE
\DeclareRobustCommand{\DIFaddendFL}{\DIFOaddendFL \let\includegraphics\DIFOincludegraphics} %DIF PREAMBLE
\DeclareRobustCommand{\DIFdelbeginFL}{\DIFOdelbeginFL \let\includegraphics\DIFdelincludegraphics} %DIF PREAMBLE
\DeclareRobustCommand{\DIFdelendFL}{\DIFOaddendFL \let\includegraphics\DIFOincludegraphics} %DIF PREAMBLE
%DIF COLORLISTINGS PREAMBLE %DIF PREAMBLE
\RequirePackage{listings} %DIF PREAMBLE
\RequirePackage{color} %DIF PREAMBLE
\lstdefinelanguage{DIFcode}{ %DIF PREAMBLE
%DIF DIFCODE_UNDERLINE %DIF PREAMBLE
  moredelim=[il][\color{red}\sout]{\%DIF\ <\ }, %DIF PREAMBLE
  moredelim=[il][\color{blue}\uwave]{\%DIF\ >\ } %DIF PREAMBLE
} %DIF PREAMBLE
\lstdefinestyle{DIFverbatimstyle}{ %DIF PREAMBLE
	language=DIFcode, %DIF PREAMBLE
	basicstyle=\ttfamily, %DIF PREAMBLE
	columns=fullflexible, %DIF PREAMBLE
	keepspaces=true %DIF PREAMBLE
} %DIF PREAMBLE
\lstnewenvironment{DIFverbatim}{\lstset{style=DIFverbatimstyle}}{} %DIF PREAMBLE
\lstnewenvironment{DIFverbatim*}{\lstset{style=DIFverbatimstyle,showspaces=true}}{} %DIF PREAMBLE
%DIF END PREAMBLE EXTENSION ADDED BY LATEXDIFF

\begin{document}

%
   \title{Modeling the track of the GD-1 stellar stream inside a host with a fermionic dark matter core-halo distribution}

   % \subtitle{I. Overviewing the $\kappa$-mechanism}

   \author{Martín F. Mestre\inst{1,2}\fnmsep\thanks{
         \email{mmestre@fcaglp.unlp.edu.ar}}\orcidlink{0009-0001-9329-5260}
      \and
      Carlos R. Argüelles\inst{1,3}\orcidlink{0000-0002-5862-8840}
      \and
      Daniel D. Carpintero\inst{1,2}
      \and
      Valentina Crespi\inst{2}
      \and
      Andreas Krut\inst{3}
   }

   \institute{Instituto de Astrof{\'i}sica de La Plata (CONICET-UNLP), Paseo del
               Bosque S/N, La Plata (1900), Buenos Aires, Argentina\\
               \and
              Facultad de Ciencias Astron{\'o}micas y Geof{\'i}sicas de La Plata (UNLP),
              Paseo del Bosque S/N, La Plata (1900), Buenos Aires, Argentina\\
              \and
              ICRANet, Piazza della Repubblica 10, 65122 Pescara, Italy\\
             }

   \date{}

% \abstract{}{}{}{}{}
% 5 {} token are mandatory

\DIFdelbegin %DIFDELCMD < \abstract{
%DIFDELCMD <     % context heading (optional), leave it empty if necessary
%DIFDELCMD <     % Stellar streams are a consequence of the tidal forces produced by a host galaxy on its satellites. Since the self-gravity of cold stellar streams can be considered negligible, they constitute excellent probes of the gravitational potential of the host galaxy. For this reason, some Milky Way (MW) stellar streams have been used to put constraints on the dark matter (DM) mass and shape, allowing to test different DM halo distributions.
%DIFDELCMD <     Traditional studies of stellar streams typically involve phenomenological $\Lambda$CDM halos or \LEt{***A&A discourages italics for emphasis or Latin expressions. Please change all your italics for emphasis or Latin expressions to upright throughout, including in the main text. I'll not highlight this again to avoid cluttering the ms***} \textit{ad hoc} dark matter (DM) profiles with different degrees of triaxiality, which preclude us from gaining insights into the nature and mass of the DM particles. Recently, the maximum entropy principle of halo formation has been applied to provide a DM halo model that incorporates the fermionic (quantum) nature of the particles while leading to DM profiles that depend on the fermion mass. These profiles develop a more general \textit{dense core} - \textit{diluted halo} morphology that can explain the Galactic rotation curve, while the degenerate fermion core can mimic the central massive black hole (BH).% (i.e. globular clusters and dwarf spheroidals)
%DIFDELCMD < }
%DIFDELCMD <    %%%
%DIF <  whose parameters are empirically determined
  %DIF <  for which analytical DM distributions (NFW, logarithmic, etc) were used.
  %DIF <  But there are also non-empirical DM distributions, among them those
  %DIF <   deduced from first principles by maximizing  the coarse-grained entropy for self-gravitating particles, so called MEPP, for Maximum Entropy Production Principle. One example is the family of Relativistic Fermionic King  (RFK) distributions, characterized by a dense core of mass $\approx4.1\times10^6 M_\odot$ (that can mimick a black hole that reproduces the orbits of the S-stars at Sag A*) and an extended halo with the plasticity of a Burkert profile.
\DIFdelend \DIFaddbegin \abstract{
    Traditional studies of stellar streams typically involve phenomenological $\Lambda$CDM halos or ad hoc dark matter (DM) profiles with different degrees of triaxiality, which preclude us from gaining insights into the nature and mass of the DM particles. Recently, the maximum entropy principle of halo formation has been applied to provide a DM halo model that incorporates the fermionic (quantum) nature of the particles while leading to DM profiles that depend on the fermion mass. These profiles develop a more general ``dense core - diluted halo'' morphology that can explain the Galactic rotation curve, while the degenerate fermion core can mimic the central massive black hole (BH).
}
\DIFaddend {
    % aims heading (mandatory)
    We model the GD-1 stellar stream using a spherical \DIFdelbegin \textit{\DIFdel{core}}%DIFAUXCMD
\DIFdel{-}\textit{\DIFdel{halo}} %DIFAUXCMD
\DIFdelend \DIFaddbegin \DIFadd{core-halo }\DIFaddend DM distribution for the host that simultaneously also explains the dynamics of the S-cluster stars through its degenerate fermion core without a central BH.
}
%DIF <  is in tune with previous fits to the S-stars.
  %DIF <  RFK
{
    % methods heading (mandatory)
    We used two optimization algorithms in order to fit both the initial conditions of the stream orbit and the fermionic model. We modeled the baryonic potential with a bulge and two disks (thin and thick) with fixed parameters according to the recent literature. The stream observables were \DIFdelbegin \DIFdel{6D }\DIFdelend \DIFaddbegin \DIFadd{5D }\DIFaddend phase-space data from the \DIFdelbegin \textit{\DIFdel{Gaia}} %DIFAUXCMD
\DIFdelend \DIFaddbegin \DIFadd{Gaia }\DIFaddend DR2 survey.
}
%DIF <  halo
{
    % results heading (mandatory)
    We were able to find good fits for both the GD-1 stream and the S-stars for a family of fermionic \DIFdelbegin \textit{\DIFdel{core}}%DIFAUXCMD
\DIFdel{-}\textit{\DIFdel{halo}} %DIFAUXCMD
\DIFdelend \DIFaddbegin \DIFadd{core-halo }\DIFaddend profiles parameterized by the fermion mass. The particle masses are constrained in the range $\SI{56}{\kilo\eV\per c^2}$, with a corresponding DM core of $\sim \num{E3}$ Schwarzschild radii, to $\SI{360}{\kilo\eV\per c^2}$, which corresponds to the most compact core of $5$ Schwarzschild radii prior to the gravitational collapse into a BH of about $\SI{4E6}{\Msun}$.
%DIF < The DM configurations have a virial radius of $\SI{27}{\kilo\parsec}$ and a virial mass of $\SI{1.4E11}{\Msun}$, the latter being at $2\sigma$ from a previous Milky Way DM halo mass estimate using the Sagittarius stream. The circular velocity at the position of the Sun, $v_{\mathrm{c}}(R_\odot)$, is not assumed in the model but it is an outcome; we obtained $v_{\mathrm{c}}(R_\odot) \approx \SI{244}{\kilo\metre\per\second}$ in agreement with recent independent estimates.
}
%DIF <   to $\SI{360}{\kilo\eV\per c^2}$, corresponding respectively to DM core radii going from $\sim \num{E3}$ to 5 Schwarzschild radii and a DM core mass of about $\SI{4E6}{\Msun}$. For smaller and larger values of the fermion mass, there was no solution that simultaneously fitted both the GD-1 stream and the S-stars
   %DIF <  distance
   %DIF <  it is a result of it
   %DIF <  for our solutions,
{
    % conclusions heading (optional), leave it empty if necessary
    This work provides evidence that the fermionic profile is a reliable model for the massive central object and for the DM of the Galaxy. Remarkably, this model predicts a total \DIFdelbegin \DIFdel{Milky Way }\DIFdelend \DIFaddbegin \DIFadd{MW }\DIFaddend mass of $2.3\times10^{11} M_{\odot}$, which agrees with recent mass estimates obtained from \DIFdelbegin \textit{\DIFdel{Gaia}} %DIFAUXCMD
\DIFdelend \DIFaddbegin \DIFadd{Gaia }\DIFaddend DR3 rotation curves (\DIFdelbegin \textit{\DIFdel{Gaia}} %DIFAUXCMD
\DIFdelend \DIFaddbegin \DIFadd{Gaia }\DIFaddend RC). In summary, with one single fermionic model for the DM distribution of the \DIFdelbegin \DIFdel{Milky Way}\DIFdelend \DIFaddbegin \DIFadd{MW}\DIFaddend , we obtain a good fit on three totally different distance scales of the Galaxy: $\sim 10^{-6}$ kpc (central, S-stars), $\sim14$ kpc~(middle, GD-1), and $\sim 30$ kpc~(boundary, \DIFdelbegin \textit{\DIFdel{Gaia}} %DIFAUXCMD
\DIFdelend \DIFaddbegin \DIFadd{Gaia }\DIFaddend RC mass estimate).
}
%DIF <  RFK
%DIF <  black hole
%DIF <  Besides
%DIF <  implies

   \keywords{Galaxy: kinematics and dynamics --
             Galaxy: halo --
             dark matter
               }
   \titlerunning{The GD-1 stream inside a fermionic dark matter halo}

   \maketitle
%________________________________________________________________

\section{Introduction}
The gravitational interaction between a galaxy and a satellite (a dwarf galaxy or a globular cluster) modifies both systems: The host galaxy strips stars from the satellite at a rate
that depends on their density profiles and on the orbit,
and the density profile of the host shows reorganized matter \DIFaddbegin \DIFadd{(due to dynamical friction) }\DIFaddend in the vicinity of the satellite orbit\DIFdelbegin \DIFdel{, which causes the profile to develop a wake due to the dynamical friction}\DIFdelend .
The ensemble of stars that are tidally stripped from the satellite constitute the so-called stellar stream (also known as tidal stream)\DIFdelbegin \DIFdel{; examples of these stars}%DIFDELCMD < \LEt{***or should this be "streams"? Please check, your use of the pronoun was ambiguous***} %%%
\DIFdelend \DIFaddbegin \DIFadd{. Examples of streams }\DIFaddend have been detected in the Milky Way (MW), in Andromeda, and in the Local Volume~\citep{Martinez-Delgado_2010}.

Stellar streams currently constitute one of the main MW observables that are related to the dynamics, together with other baryonic observables such as the Galaxy rotation curve, the radial surface density profile of the disk, and the vertical density profile of the disk at the solar radius.
Tidal streams probe the acceleration field produced by the Galaxy \DIFdelbegin \DIFdel{~}\DIFdelend \citep{1999ApJ...512L.109J,1999A&A...348L..49Z,2009ApJ...703L..67L,2013MNRAS.436.2386L,2016ASSL..420..169J,Ibata_2016,2017ApJ...842..120I,2017A&A...603A..65T,2021MNRAS.502.4170R} as well as its formation history, according to \DIFdelbegin \DIFdel{~}\DIFdelend \citet{1999Natur.402...53H}, \citet{2020ARA&A..58..205H}, \citet{2022A&A...666A..64R}, and \citet{2023arXiv230708730C}, among others.
A stellar stream could be the closest realization of a galactic orbit that can be observed in nature.
Nevertheless, the larger the progenitor, the greater the discrepancy between its orbit and the stream phase-space configuration. The taxonomy of streams is very rich \citep{2015MNRAS.450..575A} because different gravitational configurations occur, from almost \DIFdelbegin \DIFdel{one-dimensional }\DIFdelend \DIFaddbegin \DIFadd{1D }\DIFaddend streams whose progenitor is a small globular cluster, to wide shells produced by the partially radial sinking of a large progenitor into the MW center. In addition, an accreted stream whose progenitor is a globular cluster orbiting a MW satellite galaxy is theoretically possible, giving rise to stellar cocoons around the
stream track \citep{2018ApJ...861...69C,2019ApJ...881..106M,2021MNRAS.501..179M,2021ApJ...911L..32G,2022MNRAS.511.2339Q}.
Moreover, perturbations of the streams due to dark matter subhalos are theoretically possible as well. They form off-track features such as the detected spur~\citep{Price-Whelan_2018} in
the GD-1 stream~\citep{Grillmair_2006}.

Stream data together with other measurements from baryonic tracers can substantiate claims about an unknown aspect of the gravitational field: Whether it needs to be modeled with a dark matter (DM) halo component or with modified Newtonian dynamics (MoND) theory.
For example, within the DM paradigm, spatial densities might have different profiles
(e.g., spherical, axisymmetric, or triaxial) depending on the number of conserved components of the angular momentum, which also influences the stream properties \citep{2013ApJ...773L...4V,2016MNRAS.455.1079P,2020MNRAS.492.4398M}.
In particular, \citet{2019MNRAS.486.2995M} \DIFdelbegin \DIFdel{have constrained the MW }%DIFDELCMD < \LEt{***you just introduced and used an abbreviation for this above. Please decide whether you wish to spell out or abbreviate and then change throughout for consistency. This applies to all abbreviations and acronyms: they are to be introduced at first occurrence in the main text and are then to be used without reintroduction throughout for consistency, including all figure captions and tables. Please make sure you check and change this as required before you upload your final version. I'll not highlight this again to avoid cluttering the ms***}%%%
\DIFdelend \DIFaddbegin \DIFadd{constrain the MW }\DIFaddend dark matter halo shape using \DIFdelbegin \textit{\DIFdel{Gaia}} %DIFAUXCMD
\DIFdelend \DIFaddbegin \DIFadd{Gaia }\DIFaddend DR2 data of the GD-1 stream \DIFdelbegin \DIFdel{, }\DIFdelend assuming an axisymmetric generalization of the \DIFdelbegin %DIFDELCMD < \LEt{***please introduce and provide the relevant reference***}%%%
\DIFdel{NFW profile and obtaining }\DIFdelend \DIFaddbegin \DIFadd{NFW \mbox{%DIFAUXCMD
\citep{1996ApJ...462..563N} }\hskip0pt%DIFAUXCMD
profile and obtain }\DIFaddend a flattening halo parameter of $q=0.82^{+0.25}_{-0.13}$, which is compatible with spherical symmetry.
Moreover, \cite{2023MNRAS.524.2124P} \DIFdelbegin \DIFdel{were able to }\DIFdelend measure the oblateness in an axisymmetric generalization of the NFW profile using three stellar streams: NGC~3201, M68, and Palomar~5. They \DIFdelbegin \DIFdel{obtained
}\DIFdelend \DIFaddbegin \DIFadd{obtain
}\DIFaddend consistency with a spherical halo.

In addition to the above traditional halo models, which arise from $\Lambda$CDM cosmologies or from various ad hoc symmetry considerations, some DM profiles take the quantum nature of the DM candidate into account (see, e.g., \cite{2014NatPh..10..496S} for bosonic profiles that are composed of axion-like particles, and \cite{2015MNRAS.451..622R,2015PhRvD..92l3527C,2021MNRAS.502.4227A} for fermionic profiles that are typically composed of sterile neutrinos). A relevant aspect of this type of profiles is the source of quantum pressure acting in the innermost regions of the halos: While in the boson case, the profiles develop a highly dense soliton, in the fermion case, the profiles develop a degenerate compact core surrounded by a more diluted halo that is self-bounded in radius. In this work, we focus on the latter. A \DIFdelbegin \emph{\DIFdel{dense~core~--~diluted~halo}} %DIFAUXCMD
\DIFdelend \DIFaddbegin \DIFadd{dense core - diluted halo }\DIFaddend fermionic DM profile like this was obtained from the fully relativistic Ruffini-Argüelles-Rueda (RAR) model, which was successfully applied to different galaxy types \citep{2019PDU....24..278A,2023ApJ...945....1K}, including the \DIFdelbegin \DIFdel{Milky Way }\DIFdelend \DIFaddbegin \DIFadd{MW }\DIFaddend from Sagittarius~A* until the entire halo \citep{arguelles_novel_2018,2020A&A...641A..34B,2021MNRAS.505L..64B,2022MNRAS.511L..35A}.
This type of model built in terms of a Fermi-Dirac-like phase-space distribution function (including central degeneracy and cutoff in the particle energy) is also known as the relativistic fermionic King (RFK) model \citep{2022PhRvD.106d3538C}.
% DM microphysics affects the overall structure of DM halos. That is, our ignorance about its potential (quantum) effects through the innermost regions of the halos and their stability, may represent a hole in our knowledge of these systems. the use of classical point masses as the building blocks of matter (without considering the quantum nature of the particles) in traditional N-body simulations within the LCDM paradigm does not allow for any new possible sources of quantum-pressure in DM halos (e.g. as recently shown within 3D numerical simulations for bosons~\refcite{Schive} or fermions ~\refcite{Arguelles1}). The arising of these quantum effects at the center of equilibrium systems of self-gravitating quantum particles (either bosons or fermions) is a long-known physical phenomenon \cite{RB}

%In the recent literature many types of DM candidates have been analyzed, like axions, bosons and fermions.
%In this work we will work with the relativistic fermionic King model defined in~\citet{2020EPJP..135..290C}, as implemented in the RAR model~\citep{arguelles_novel_2018}. {\color{cyan}...hasta aca.}

We model the \DIFdelbegin %DIFDELCMD < \LEt{***you spelled out "one-dimensional" above. Please decide whether you wish to spell out or abbreviate and change throughout for consistency***}%%%
\DIFdel{6D }\DIFdelend \DIFaddbegin \DIFadd{5D }\DIFaddend track of the GD-1 stellar stream inside an MW with a fermionic dark matter core-halo distribution. At the same time, we aim to explain the dynamics of the S-cluster stars through the high-density fermion core and without assuming a central BH. Finally, we compare the virial mass of the Galaxy as predicted by the fermionic model with that obtained from recent \DIFdelbegin \textit{\DIFdel{Gaia}} %DIFAUXCMD
\DIFdelend \DIFaddbegin \DIFadd{Gaia }\DIFaddend DR3 rotation-curve data. In Sec.~\ref{sec:methodology} we explain the \DIFdelbegin \DIFdel{method}\DIFdelend \DIFaddbegin \DIFadd{methods}\DIFaddend ; in Sec.~\ref{sec:results} we present the best-fit results, and in Sec.~\ref{sec:conclusions} we conclude.

\section{\DIFdelbegin \DIFdel{Method}\DIFdelend \DIFaddbegin \DIFadd{Methods}\DIFaddend }
\label{sec:methodology}
In this section, we explain the observables and the methods we used in this research.
The exact pipeline applied in order to obtain the results and plots of this paper
can be found at the \DIFdelbegin \DIFdel{following GitHub repository:
}%DIFDELCMD < \url{https://github.com/martinmestre/stream-fit/blob/main/pipeline_paper/}%%%
\DIFdelend \DIFaddbegin \DIFadd{corresponding GitHub repository}\footnote{
\url{https://github.com/martinmestre/stream-fit/blob/main/pipeline_paper/}}\DIFaddend .

\subsection{Observables and assumed measurements}
\label{sec:observables}

The main observables that we used were computed with the polynomial fits found by
\citet{Ibata_2020} for the \mbox{GD-1} stream using astrometric (\DIFdelbegin \textit{\DIFdel{Gaia}} %DIFAUXCMD
\DIFdelend \DIFaddbegin \DIFadd{Gaia }\DIFaddend DR2) and high-precision spectroscopic datasets, together with the analysis of the {\sc streamfinder} algorithm.
The polynomials are the following:
\begin{align}
   \label{Ibata_polyn}
   \phi_2  &= 0.008367\phi_1^3-0.05332\phi_1^2-0.07739\phi_1-0.02007, \\
   \tilde{\mu}_\alpha &= 3.794\phi_1^3+9.467\phi_1^2+1.615\phi_1-7.844,\\
   \mu_\delta &= -1.225\phi_1^3+8.313\phi_1^2+18.68\phi_1-3.95,\\
   v_h &=  90.68\phi_1^3+204.5\phi_1^2-254.2\phi_1-261.5,
\end{align}
with $\phi_1$ and $\phi_2$ in radians, $\tilde{\mu}_\alpha=\mu_\alpha \cos \delta$ and $\mu_\delta$ in mas yr$^{-1}$, and $v_h$ in km s$^{-1}$. These quantities correspond to the longitude and latitude in the GD-1 celestial frame of reference~\citep{Koposov_2010}, the proper motion in right ascension and declination, and the heliocentric radial velocity, respectively.
The domain of the polynomials was limited to $-90^\circ <\phi_1<10^\circ$.
To obtain our observable data set, we sampled the domain with 100 equidistant points ($\phi_1^{(i)},\, i=1,\dots,100$)
and evaluated the polynomials at these points, thus obtaining the sets $\phi_1^{(i)}$, $\phi_2^{(i)}$, $\tilde{\mu}_\alpha^{(i)}$, $\mu_\delta^{(i)}$, and $v_h^{(i)}$. We did not include among the observables in Eq.~(\ref{Ibata_polyn}) the photometric heliocentric distance $D$, which is justified by the posterior analysis performed in Sec.~\ref{sec:fitting}.
In one of the experiments, we considered an observable of a different nature,
the core mass ($M_{\rm{core}}$), which is defined as the mass enclosed at the radius when the circular velocity reaches its first maximum.
The constraint for the mass of the core of the distribution was assumed to be $M_{\rm core } = \SI{3.5E6}{\Msun}$, in agreement with~\citet{2020A&A...641A..34B,2021MNRAS.505L..64B}, and \citet{2022MNRAS.511L..35A}.
The core radius does not include the entire mass inside the innermost S-star pericenter we considered (S2) because the first maximum of the circular velocity corresponds to a shorter distance at which the core density-region is still falling. In Sec. \ref{sec:results} we obtain a DM mass inside the S2 star pericenter of $M(r_{peri-\rm{S2}}) = 4.03\times 10^6 M_\odot$, which agrees very well with the Schwarzschild BH mass constraints of $4.1\times 10^6 M_\odot$ and $3.97\times 10^6 M_\odot$ obtained from the same S2 star in \cite{2018A&A...615L..15G} and \cite{2019Sci...365..664D}, respectively.
The parameters assumed in this paper are the
galactocentric distance of the Sun $R_\odot=\SI{8.122}{\kilo\parsec}$ \citep{2018A&A...615L..15G} and the peculiar solar
velocity\footnote{
We adopted a Cartesian reference frame $(X, Y, Z)$ with
corresponding velocities $(U, V, W)$ in which the $X$ and $U$ axes
point from the Galactic center in the opposite direction
of the Sun; $Y$ and $V$ point in the direction of the Galactic rotation at the
location of the Sun; and $Z$ and $W$ point
toward the Galactic north pole. This is the same right-handed reference system as is adopted by Astropy.} $\vec{v}_{\odot p} = (\SI{11.1}{\kilo\metre\per\second}, \SI{12.24}{\kilo\metre\per\second}, \SI{7.25}{\kilo\metre\per\second})$ \citep{Shonrich}.


\subsection{Fermionic dark matter halo model}
\label{sec:fermionicDM}
Our fermionic DM model is a spherical and isotropic distribution of fermions at finite temperature in hydrostatic equilibrium, subject to the laws of General Relativity (GR), that is, the T.O.V. equation complemented with the Tolman and Klein thermodynamics equilibrium conditions and the particle energy conservation along a geodesic, as defined in~\cite{arguelles_novel_2018}, while using a notation from~\cite{2020EPJP..135..290C} as detailed below.

We started with a spherically symmetric metric, which is defined as
\begin{equation}
    \label{metric}
    \mathrm{d}s^2 = g_{00}(r)\mathrm{d}t^2 + g_{11}(r)\mathrm{d}r^2 -r^2\mathrm{d}\vartheta^2 -r^2\sin\vartheta \mathrm{d}\varphi^2,
\end{equation}
with $g_{00}(r) = \mathrm{e}^{\nu(r)}c^2$ and $g_{11}(r) = -\mathrm{e}^{\lambda(r)}$,
where $c$ is the speed of light, $t$ stands for the time, ($r$, $\vartheta$, $\varphi$) are spherical coordinates, and $\nu$ and $\lambda$ are metric exponents whose radial dependence is computed below.

Our first differential equation is that of the mass versus radius for a spherical system of density $\rho$,
 \begin{equation}
    \label{mass_def}
    \frac{\mathrm{d}M}{\mathrm{d}r} = 4\pi r^2 \rho(r),
 \end{equation}
from which we obtained the enclosed mass $M(r)$ at a given radius $r$ by simple integration,
\begin{equation}
 M(r)=4\pi\int_0^r r'^2 \rho(r') {\mathrm{d}}r'.
\end{equation}
From the Einstein equations, the relation between $M(r)$
and the metric exponent $\lambda$ can be found,
\begin{equation}
    \mathrm{e}^{-\lambda(r)}=1-\frac{2G}{c^2}\frac{M(r)}{r},
\end{equation}
where $G$ is the gravitational constant.

Additionally, the following version of the T.O.V. equation for the metric exponent $\nu$ can be deduced:
\begin{align}
   \label{tov}
    \frac{\mathrm{d}\nu}{\mathrm{d}r}= \frac{1+ \frac{\displaystyle 4\pi r^3 P(r)}{\displaystyle M(r)c^2}}{r\left(\frac{\displaystyle rc^2}{\displaystyle 2GM(r)} -1\right)},
\end{align}
where $P$ is the pressure.
The two quantities $\rho$ and $P$ are defined as the following integrals over momentum space,
\begin{align}
     \label{dens_press}
      \rho(r)&=\int \frac{E(p)}{c^2}~f(r,p)\mathrm{d}\boldsymbol{p},\\
      P(r)&=\frac{1}{3}\int p\frac{dE(p)}{dp}~f(r,p)\mathrm{d}\boldsymbol{p}=
                \frac{1}{3}\int \frac{p^2c^2}{E(p)}~f(r,p)\mathrm{d}\boldsymbol{p},
\end{align}
where $\boldsymbol{p}$ is the spatial momentum  vector, $p$ is its norm, $E(p)=\sqrt{p^2c^2+m^2c^4}$ is the total energy of a particle of mass $m$, and $f$ is the phase-space distribution of the system, given by a Fermi-Dirac distribution with an energy cutoff. This number density $f$ can be obtained from a maximum entropy principle computed from a kinetic theory that includes self-gravity and violent relaxation, as shown in \citealp{2004PhyA..332...89C}~\citep[for a review see also][]{2022PhyA..60628089C}, and was recently applied to a vast sample of disk galaxies in \cite{2023ApJ...945....1K}. It can be expressed as
\begin{equation}
f(r,p)=
    \frac{g}{h^3}
      \frac{1-\mathrm{e}^{[E(p)-E_\mathrm{c}(r)]/kT(r)}}{1+\mathrm{e}^{[E(p)-\mu(r)]/kT(r)}}\quad\mathrm{if}\quad E(p) \leq E_\mathrm{c}(r),\\
\end{equation}
and $f(r,p)=0$ otherwise,
where $k$ is the Boltzmann constant, $h$ is the Planck constant, $g=2s+1$ is the spin multiplicity of quantum states, with $s=1/2$, and the following local quantities were used: the chemical potential $\mu(r)$, the cutoff energy $E_\mathrm{c}(r)$,  and the effective temperature $T(r)$.
The coefficient $g/h^3$ is the maximum accessible value of the distribution function fixed by the Pauli exclusion principle.

The above equations were complemented with two thermodynamic equilibrium conditions given
by~\citet{PhysRev.35.904} and~\citet{RevModPhys.21.531}, together with the condition of energy conservation
along the geodesic given in~\citet{1989A&A...221....4M},
\begin{equation}
    \label{tke}
   \frac{1}{T}\frac{\mathrm{d}T}{\mathrm{d}r}=\frac{1}{\mu}\frac{\mathrm{d}\mu}{\mathrm{d}r}=
   \frac{1}{E_\mathrm{c}}\frac{\mathrm{d} E_\mathrm{c}}{\mathrm{d}r}=-\frac{1}{2}\frac{\mathrm{d}\nu}{\mathrm{d}r}.
\end{equation}

Thus, we built a system of five differential equations given
by Eqs.~(\ref{mass_def}), (\ref{tov}), and (\ref{tke}), with initial conditions
at the center of the distribution $M(0)=0$, $\nu(0)=0$, $T(0)=T_0$,
$\mu(0)=\mu_0$, and $E_c(0)=E_{c0}$.
The differential equations do not depend on $\nu$, but on its radial
derivative, so that the system can be solved starting with an arbitrary initial value $\nu(0)=0$, adding a finite value later, namely $\nu_0$, in such a way that the solution satisfies the condition of continuity with the Schwarzschild metric at the border of the fermionic distribution. In Appendix~\ref{app:numerical} we solve the system of equations numerically. Following~\cite{arguelles_novel_2018}, we use adimensional versions of the initial conditions throughout,
\begin{align}
    \theta_0 &= \frac{\mu_0 - mc^2}{kT_0},\nonumber\\
    W_0 &= \frac{E_{c0} - mc^2}{kT_0},\nonumber\\
    \beta_0 &= \frac{kT_0}{mc^2}.\label{rar_params}
\end{align}
We subtracted the rest-energy in the first two cases.

Although our fermionic system is univocally determined by the four free parameters $m$, $\theta_0$, $W_0$, and $\beta_0$, in some convenient situations, we use $\omega_0=W_0-\theta_0$ instead of $W_0$.




\subsection{\DIFdelbegin \DIFdel{Milky Way }\DIFdelend \DIFaddbegin \DIFadd{MW }\DIFaddend and stream models}
\label{sec:MW_stream_models}

We modeled our Galaxy by combining the fermionic dark halo described above, whose parameters are determined in this work,with a fixed baryonic component identical to that in Model I of \citet{2017A&A...598A..66P}. We call this full Galaxy model Fermionic-MW.

In addition to this model and for qualitative comparative purposes, we used the Galactic model fit by \citet{2019MNRAS.486.2995M}, which is the {\texttt{MWPotential2014}} model with an axisymmetric NFW profile. This model was evaluated by means of the {\it Galpy} package~\citep{2015ApJS..216...29B}. A robust statistical comparison between the RAR and other models will be performed in a future paper, but here, we intend to verify that our GD-1 fit agrees with the latest fit performed in the literature. This fitted model uses a circular velocity at the position of the Sun $v_\mathrm{c}(R_\odot)=244 \pm 4$ km s$^{-1}$ and a $z-$flattening of the DM density distribution  $q_\rho=0.82^{+0.25}_{-0.13}$. We call this Galaxy model NFW-MW.
% some intermediary tasks

GD-1 is a dynamically cold stream, and its stars are highly correlated, although
its progenitor has not been detected with certainty~\citep{10.1093/mnras/sty677,Price-Whelan_2018,10.1093/mnras/sty1338}.
Its almost one-dimensional distribution in \DIFdelbegin \DIFdel{phase space }\DIFdelend \DIFaddbegin \DIFadd{phase-space }\DIFaddend could be well approximated with the orbit of a theoretical progenitor, however, as previously reported by~\cite{2019MNRAS.486.2995M,Price-Whelan_2018} and \cite{2010ApJ...712..260K}.

The initial conditions of the orbit were given in the spherical equatorial coordinates of the ICRS frame: right ascension $\alpha$, declination $\delta$, $D$, $\tilde{\mu}_\alpha$, $\mu_\delta$, and $v_h$. The code uses the {\it Astropy} ecosystem~\citep{astropy:2022, astropy:2018, astropy:2013} in order to transform these initial conditions to galactocentric coordinates assuming a galactocentric reference frame with the Sun at the position $\vec{x}_\odot=(-R_\odot,0,0)$ and the solar velocity given by the sum of the circular velocity at the position of the Sun and the peculiar solar velocity: $\vec{v}_\odot = \vec{v}_{\odot p} + (0, v_{\rm{c}}(R_\odot), 0)$. The circular velocity depends on the model and position and is given by \begin{equation}
\label{circ_vel}
    \begin{split}
       v_{\rm{c}}^2(R_\odot) &= R_\odot|\nabla \Phi(\vec{x})|_{\vec{x}=\vec{x}_\odot}\\
       &= R_\odot|\nabla \Phi_{\rm{B}}(\vec{x})|_{\vec{x}=\vec{x}_\odot} + G \frac{M_{\rm{DM}}(R_\odot)}{R_\odot},
    \end{split}
\end{equation} where $\Phi$ is the total potential, $\Phi_B$ is the potential generated by the three baryonic components, and $M_{\rm{DM}}$ is the enclosed fermionic DM mass. In the last term, we used the spherical symmetry of the DM distribution in order to relate the acceleration with the enclosed mass. For the NFW-MW model, the circular velocity was computed using the gradient of the total potential, that is, the first line of Eq.~(\ref{circ_vel}), performed with the {\it Galpy} code.

We integrated the orbit forward and backward in time during a time interval of $\Delta t=\SI{0.2}{\giga\year}$, starting in both cases from a given initial condition for the progenitor. In the next sections we explain how these initial conditions were chosen for some simulations and fitted for others. The integrator we used was an eighth-order Runge-Kutta (\texttt{DOP853} called from the {\it SciPy} \texttt{solve\_ivp} function) with relative and absolute tolerance parameters given by \texttt{rtol}$=\num{5E-14}$ and \texttt{atol}$=\num{0.5E-14}$, respectively.

The resulting orbit was successively transformed into the ICRS and then into the GD-1 coordinate frames. For the latter, we used the \texttt{GD1Koposov10} class defined in the \texttt{Gala} package~\citep{gala,adrian_price_whelan_2020_4159870}, which uses the transformation matrix defined by \citet{Koposov_2010}.
After these transformations, we obtained the orbit expressed in the observable variables $\phi_1$, $\phi_2$, $\tilde{\mu}_\alpha$, $\mu_\delta$, and $v_h$. Finally, we computed these variables at the points $\phi_1^{(i)}$ by interpolation. These values, together with the observed GD-1 data defined in Sec.~\ref{sec:observables}, were used to evaluate the following stream function:
\begin{align}
   \label{chi2:stream}
   \chi^2_{\rm{stream}} &= \chi^2_{\phi_2} + \chi^2_{\tilde{\mu}_\alpha}+ \chi^2_{\mu_\delta}+\chi^2_{v_h}\\
   \label{chi2:component}
   \chi^2_{\eta} &= \frac{1}{\sigma_\eta^2}\sum_{i=1}^{100} \parantheses{\eta^{(i)}-\eta(\phi_1^{(i)})}^2,
\end{align}
where
$\eta\in\braces{\phi_2, v_h, \tilde{\mu}_\alpha, \mu_\delta}$, and $\sigma_\eta$ are the corresponding dispersions of the stream data points estimated by inspection of Figs.~1, 3, and 4 of~\citet{Ibata_2020},
$\sigma_{\phi_2}=0\fdg 5$,
$\sigma_{v_h} = 10$~km s$^{-1}$, and
$\sigma_{\tilde{\mu}_\alpha}= \sigma_{\mu_\delta}= 2$~mas yr$^{-1}$.
Thus, $\chi^2_{\rm{stream}}$ measures the departure of the model from the observed stream.

For some fits, we also considered the departure of the model from a dark mass constraint in the core
of the distribution,
\begin{equation}
   \label{chi2:core}
\chi^2_{\rm{core}}=\frac{(m_\mathrm{c}-M_{\rm{core}})^2}{\sigma_m^2},
\end{equation}
where the value of $M_{\rm{core}}$ was defined in Sec.~\ref{sec:observables}, $m_\mathrm{c}$ is the core mass of the model (i.e., the variable), and
$\sigma_m$ was fixed at $0.01 M_{\rm{core}}$.
For these fits, we then used the following compound function:
\begin{equation}
   \label{chi2:full}
\chi^2_{\rm{full}}=\chi^2_{\rm{stream}}+\chi^2_{\rm{core}}.
\end{equation}
We note that in order to compute the estimated core mass for each model, $m_\mathrm{c}$ in \cref{chi2:core}, we calculated the first local maximum of the circular velocity in GR. This is compactly expressed as
\begin{equation}
   \label{v_circ_dm}
   V_{\rm{circ, DM}}(r)= c\left(\frac{r}{2}\frac{g_{00}'(r)}{g_{00}(r)}\right)^{1/2} =
                        c\left(\frac{r}{2}\frac{\mathrm{d}\nu(r)}{\mathrm{d}r}\right)^{1/2},
\end{equation}
where the velocity has components $V^i=dx^i/dt~(i=1,2,3)$ measured in a local frame that was fixed in space at a distance $r$ from the Galaxy center. From the T.O.V. \cref{sode2}, it is possible to show that far from the core, this relativistic formula for the velocity tends to the classical law, that is, $\sqrt{G~M(r)/r}$.
%It can be demonstrated that this relativistic formula
%for the circular velocity transforms into the classical law, i.e. $\sqrt{G~M(r)/r}$, when being far from the core.}

\subsection{Optimization algorithms}
\label{sec:optimization}

Our goal was to fit our Fermionic-MW model by minimizing the full $\chi^2$ function given by \cref{chi2:full} (when fitting the NFW-MW model, we only used the function given by \cref{chi2:stream}). To this end, we used two optimization algorithms that belong to the family of black box algorithms, which perform very well when the function to be minimized presents many relative minimums or when the landscape is complex, as in our case.
% chi2:stream,chi2:component,chi2:core,
One algorithm is an implementation of a differential evolution algorithm, which is a metaheuristic algorithm that consists of finding the solution of an optimization problem by iteratively updating generations of candidate solutions until a certain tolerance is met. Generally, a few best candidates from each generation survive in order to create the descendants, that is, the next generation, by making stochastic combinations from them. We used the {\it SciPy} implementation of this algorithm, called \texttt{optimize.differential\_evolution} algorithm, with the metaparemeters given by \texttt{strategy}=\texttt{"best2bin"}, \texttt{maxiter}=200, \texttt{popsize}=200, \texttt{tol}=$\num{5E-8}$ and \texttt{atol}=0, unless otherwise stated. This method can be run in parallel with shared memory.


The other algorithm is an implementation of the mesh adaptive direct search (MADS) algorithm called
\texttt{NOMAD}~\citep{audet2021nomad}. \cite{MADS_2006} explained the method in detail. The {\sc Julia}~\citep{bezanson2017julia}
wrapper of this algorithm, \href{https://bbopt.github.io/NOMAD.jl/stable/}{\it NOMAD.jl},
was used through the package \href{https://docs.sciml.ai/Optimization/stable/}{\it Optimization.jl}.
We used default values of all the metaparameters, except for \texttt{maxiters}=700.

\section{Results}
\label{sec:results}

\subsection{Fitting the Fermionic-MW model}
\label{sec:fitting}
We fit both the Fermionic-MW model parameters and the initial conditions (IC) of the orbit of the progenitor through four steps that consisted of (i) obtaining an order zero value of the orbit IC using the
NFW-MW model, (ii) fitting the Fermionic-MW parameters with fixed IC values, (iii) polishing the IC values with fixed Fermionic-MW potential, and (iv) polishing the Fermionic-MW parametes with fixed IC values.

As a first step, we searched for a provisional but good enough set of IC that was able to reproduce the orbit in our
Fermionic-MW model. To this end, we fit the initial conditions in the fixed NFW-MW model by using the $\chi^2_{\rm{stream}}$ function as defined in \cref{chi2:stream}
and the differential evolution algorithm specified in Sec.~\ref{sec:optimization}. Since the optimization algorithm
needs bounds for the variables, we used boxes centered near the midpoint
of the observable data, that is, $\eta^{(51)}$, which corresponds to $\alpha=149\fdg 24$, $\delta=36\fdg 61$, $\tilde{\mu}_\alpha=-5.70$ mas yr$^{-1}$, $\mu_\delta=-12.48$ mas yr$^{-1}$, and $v_h=-18.81$ km s$^{-1}$
plus $D=7.69$ kpc,\footnote{This value corresponds to evaluating $D(\phi_1^{(51)})$ according to Eqn.~(\ref{phot_dist}).} and with sides of length equal to the absolute values of the variables at their centers (except for $\alpha$, where we used a side of length $\alpha/5$).

The differential evolution algorithm converged to the solution
$\alpha=149\fdg 25$, $\delta=36\fdg 59$, $D=8.01$ kpc, $\tilde{\mu}_\alpha=-5.56$ mas yr$^{-1}$, $\mu_\delta=-12.40$ mas yr$^{-1}$, and $v_h=-19.15$ km s$^{-1}$, with a value of $\chi^2_{\rm{stream}}= 11.58$.
The orbit that corresponds to these IC is displayed in the observable space in Fig.~\ref{fig:obs_astrometry}~(top: $\phi_2$; middle: $\tilde{\mu}_\alpha$, $\mu_\delta$; bottom: $v_h$) with a dotted (green) line. The solid (black) line shows the corresponding observable data $\eta$, while the shaded (gray) area demarcates the corresponding $1\sigma_\eta$ regions.
\begin{figure}
   \centering
   \includegraphics[width=\hsize]{figures/observables_position.pdf}
   \includegraphics[width=\hsize]{figures/observables_pmra.pdf}
   \includegraphics[width=\hsize]{figures/observables_pmdec.pdf}
   \includegraphics[width=\hsize]{figures/observables_helvel.pdf}
   \caption{Stream fits in observable space: Sky position (top: $\phi_2$), proper motions (middle: $\tilde{\mu}_\alpha$, $\mu_\delta$), and heliocentric velocity (bottom: $v_h$).}
   \label{fig:obs_astrometry}
\end{figure}

\begin{figure}
   \includegraphics[width=\hsize]{figures/observables_heldist.pdf}
   \caption{Photometric distance ($D$). This was not included as an observable in the stream-fitting procedure.}
   \label{fig:obs_radial}
\end{figure}
Previous works~\citep{arguelles_novel_2018,2019PDU....24..278A,2023ApJ...945....1K} have shown that the parameters of the fermionic DM model can be split into two sets for the family of fermionic DM profiles with highly degenerate cores (i.e., $\theta_0 \gtrsim 15$ \citealp{2019PDU....24..278A}). On one hand, $m$ and $\beta_0$ control the core of the distribution in the sense that for given values of $M_{\rm{core}}$ and $m$, it is possible to find a consistent value of $\beta_0$ with little influence on the other two parameters $\theta_0$ and $W_0$. This implies a partial degeneracy between $m$ and $\beta_0$.
On the other hand, for the same type of fermionic \DIFdelbegin \textit{\DIFdel{core}}%DIFAUXCMD
\DIFdel{-}\textit{\DIFdel{halo}} %DIFAUXCMD
\DIFdelend \DIFaddbegin \DIFadd{core-halo }\DIFaddend solutions with a positive central degeneracy as mentioned above, the main behavior of the distribution in the halo is determined by $\omega_0$, as reported in \cite{2019PDU....24..278A} for different galaxy types.
% and $W_0$ (or $\omega_0$) have a small effect on the core region but they
%determine
%There is also a partial
%degeneracy between these two parameters because it is the combination
%$\omega_0 = W_0-\theta_0$ which mainly controls the halo profile, as will be shown in this paper.

As a starting point for the fitting of the Fermionic-MW parameters (second step), we took the values obtained by \citet{2020A&A...641A..34B}: $(m, \beta_0, \theta_0, W_0)=(56~\rm{keV}~c^{-2}, 1.1977\times10^{-5}, 37.765, 66.3407)$, which allowed their MW model (DM+barions) to satisfy the geodesic motions of both S2 and G2 at Sagittarius~A* and the rotation curve from~\citet{sofue_rotation_2013}.
From these values, fixing $m=56~\rm{keV}~c^{-2}$ and taking the already fitted IC of the stream progenitor for the NFW-MW model, we performed a differential evolution minimization of the
$\chi^2_{\rm{full}}$ function in the window
$(\theta_0,\omega_0, \beta_0)\in [35, 40]\times[25, 30]\times[10^{-5}, 1.5\times10^{-5}]$. Using metaparameter values \texttt{maxiter}$~=~$\texttt{popsize}$~=~$300, the algorithm converged to
$(\theta_0, \omega_0, \beta_0)= (36.094, 27.368 , 1.252\times10^{-5})$, giving
$\chi^2_{\rm{stream}}= 16.190$ and  $\chi^2_{\rm{core}}= 7.676\times10^{-10}$.

The third step consisted of polishing the IC of the orbit by using the differential evolution algorithm with fixed Fermionic-MW parameters, with metaparameter values \texttt{maxiter}$~=~$\texttt{popsize}$~=~$400, which gave a result very similar to that of the FMW-MW case:
$\alpha=149\fdg 39$, $\delta=36\fdg 87$, $D=8.02$ kpc,
$\tilde{\mu}_\alpha=-5.55$ mas yr$^{-1}$, $\mu_\delta=-12.33$ mas yr$^{-1}$ and $v_h=-20.84$ km s$^{-1}$, with an improved value of $\chi^2_{\rm{stream}}= 13.59$.

The last step consisted of polishing the fermionic parameters using the second optimization algorithm described in Sec.~\ref{sec:optimization}, that is, \texttt{NOMAD}. We divided a  macroscopic orthohedron in parameter space, $(\theta_0, \omega_0, \beta_0) \in [35.8, 36.3]\times[27.0, 27.6]\times [1.2\times10^{-5}, 1.3\times10^{-5}]$, into $17^3=4913$ smaller orthoedrons. In each subregion, we performed an independent optimization in a parallel distributed scheme and searched for the parameters that minimized $\chi^2_{\rm{full}}$. Then, we selected the global minimum by comparing the results of each distributed process and obtained the following final fit parameters of the model: $(\theta_0, \omega_0, \beta_0)= (36.0704, 27.3501, 1.2527\times10^{-5})$, giving $\chi^2_{\rm{full}}= 13.53$.
The corresponding orbit is displayed in the observable space in
Fig.~\ref{fig:obs_astrometry} with a dashed (amber) line. The Fermionic-MW and NFW-MW models both fit the GD-1 stream very well.
We did not perform any statistically rigorous comparison of the two models to determine which was more consistent with the data. In a future paper, we will compute the posterior distribution of the fit parameters and will then be able to give error bounds.


For completeness, in Fig.~\ref{fig:obs_radial} we plot the heliocentric distance ($D$) using the same line types as in Fig.~\ref{fig:obs_astrometry}. The solid (black) line corresponds to the fifth-order polynomial fit in~\cite{Ibata_2020} to the photometric distance measured there,
\begin{align}
 \label{phot_dist}
    \begin{split}
        D &= -4.302\phi_1^5 - 11.54\phi_1^4 - 7.161\phi_1^3 + 5.985\phi_1^2\\
      &\phantom{=} + 8.595\phi_1+10.36.
    \end{split}
\end{align}
The two theoretical $D(\phi_1)$ curves agree with each other, but they both differ considerably with respect to the observed curve (i.e., the polynomial). Because the polynomial presents a suspicious constant value for $\phi_1 \in [-70,-30]$, we plotted the galactocentric distance versus $\phi_1$ (not shown), and we found that this latter curve presents an unphysical wobbling behavior. We therefore did not include the photometric distance as a fitting observable, as is usually done in the literature.


It is instructive to see how the value of $\chi^2_{\rm{stream}}$ is modified when the parameters $\theta_0$ and $\omega_0$ (i.e., those that control the halo) are varied. To this end, we fixed $(m, \beta_0)=(56~\rm{keV}~c^{-2}, 1.2527\times10^{-5})$ and varied $(\theta_0, \omega_0)$ in a grid spanning the rectangle $[34, 38]\times[25, 30]$.
Fig.~\ref{fig:chi2stream} shows a contour plot of this grid, colored by the value of their corresponding $\chi^2_{\rm{stream}}$; the black point corresponds to our fit solution.
The minima of the function are located in a thin valley, which shows that the solution of the fitting problem is locally degenerate along a straight line in the $(\theta_0, \omega_0)$ plane.
To show the behavior of $\chi^2_{\rm{stream}}$ around the solution, we first fit a line to the points that satisfied $\chi^2_{\rm{stream}}<50$. We
obtained $h(x)= 0.7939+0.7362x$. We then subtracted this line from $\omega_0$, and finally remade a plot of $\chi^2_{\rm{stream}}$ in the window $(\theta_0, \omega_0-h(\theta_0))\in[35, 37]\times[-0.011,0.011]$. The result is shown in Fig.~\ref{fig:chi2stream_tilted}. It is noticeable that a minimum (nondegenerate problem) indeed exists and that
the variance along $\omega_0-h(\theta_0)$ is two orders of magnitude smaller than the variance along $\theta_0$.
%
\begin{figure}
   \centering
   \includegraphics[width=\hsize]{figures/chi2stream_contourf.pdf}
   \caption{Values of $\chi^2_{\rm{stream}}$ for $(m, \beta_0)=(56~\mathrm{keV}~c^{-2}, 1.254\times10^{-5})$ in the window $(\theta_0, \omega_0)\in[34, 38]\times[25.5,28.2]$. The black point corresponds to our solution, and the gray region corresponds to $\chi^2_{\rm{stream}} >10^4$. The minima of the function are located along a thin valley.}
   \label{fig:chi2stream}
\end{figure}
\begin{figure}
   \centering
   \includegraphics[width=\hsize]{figures/chi2stream_tilted_contourf.pdf}
   \caption{Values of $\chi^2_{\rm{stream}}$ for $(m, \beta_0)=(56~\mathrm{keV}~c^{-2}, 1.254\times10^{-5})$ in
   the window $(\theta_0, \omega_0-h(\theta_0))\in[35, 37]\times[-0.011,0.011]$. The black point corresponds to our solution, and the gray region corresponds to $\chi^2_{\rm{stream}} > 500 $. }
   \label{fig:chi2stream_tilted}
\end{figure}

\subsection{Rotation curves, accelerations, and virial quantities}
We computed the resulting rotation curves for our two models and compared them with three observed rotation curves (Fig.~\ref{fig:rotcurve}).

\begin{figure}
   \centering
   \includegraphics[width=\hsize]{figures/rotation_curves.pdf}
   \caption{Rotation curves of the Fermionic-MW model (this work, in dashed amber) and the NFW-MW model (\cite{2019MNRAS.486.2995M}, in dotted green), which fit the GD-1 stream, are compared \DIFdelbeginFL \textit{\DIFdelFL{a posteriori}} %DIFAUXCMD
\DIFdelendFL \DIFaddbeginFL \DIFaddFL{a posteriori }\DIFaddendFL with different observed rotation curves (\citealp{Eilers_2019} with purple triangles,~\citealp{sofue_2020} with light blue rhombi, and \citealp{Jiao2023} with black circles). Only the Fermionic MW model can account for the GD-1 stream data and the sharp drop of the recent \DIFdelbeginFL \textit{\DIFdelFL{Gaia}} %DIFAUXCMD
\DIFdelendFL \DIFaddbeginFL \DIFaddFL{Gaia }\DIFaddendFL DR3 rotation curve.}
   \label{fig:rotcurve}
\end{figure}
%DIF <  Rotation curves of the MW models and from observations: NFW-MW model in dotted (green) line, Fermionic-MW in dashed (amber) line,~\citet{Eilers_2019} with triangles (purple),~\citet{sofue_2020} with rhombi (light-blue) and \citet{Jiao2023} with circles (black). The region where the GD-1 stream is located at present is displayed with a vertical grey band. Note that the NFW-MW model has been fitted using GD-1 data while the Fermionic-MW model used the stream and core mass constraints, but neither of the models used the rotation curves data as a constraint.


\DIFdelbegin \DIFdel{The dotted (green) line corresponds to the NFW-MW model and the dashed (amber) line to the Fermionic-MW model.
}%DIFDELCMD < \LEt{***details from the figure caption such as line styles are not to be repeated in the main text. Please remove the next sentence and check throughout for repetitions like this, which are then also to be removed***}%%%
\DIFdel{The observed rotation curves are coded according to triangles (purple)
for \mbox{%DIFAUXCMD
\citet{Eilers_2019}}\hskip0pt%DIFAUXCMD
, rhombi (light-blue) for \mbox{%DIFAUXCMD
\citet{sofue_2020} }\hskip0pt%DIFAUXCMD
and
circles (black) for \mbox{%DIFAUXCMD
\citet{Jiao2023}}\hskip0pt%DIFAUXCMD
.
}\DIFdelend In order to build a unified rotation curve, \citet{sofue_2020} computed the running average of many rotation curves resulting from different dynamical tracers according to the galactocentric distance. In the central parts of the Galaxy, the tracers used were the molecular gas and the infrared stellar motion, while in the outer parts (beyond $r\sim30 \textrm{~kpc}$), the tracers were the radial motions of satellite galaxies and globular clusters. Rotation curves resulting from Galactic disk objects were also used.
On the other hand, \citet{Eilers_2019} used a selection of \DIFdelbegin %DIFDELCMD < \LEt{***you use this abbreviation only this once. Please spell it out***}%%%
\DIFdel{RGB }\DIFdelend \DIFaddbegin \DIFadd{red giant branch }\DIFaddend stars as tracers of the disk dynamics.

It is worth noticing  that \citet{sofue_2020} assumed
($R_\odot$,~$v_c(R_\odot))=(8$ kpc, 238 km s$^{-1}$), while \citet{Eilers_2019}
assumed $R_\odot=8.122$~kpc and a solar galactocentric velocity $v_\odot$ = $(11.1, 245.8, 7.8)$~km s$^{-1}$,
with which they estimated to be $v_c(R_\odot)=229\pm0.2$ km s$^{-1}$.

It is interesting to note that the two theoretical models, Fermionic-MW and NFW-MW, both give $v_\mathrm{c}(R_\odot)\approx 244$~km s$^{-1}$ \DIFdelbegin %DIFDELCMD < \LEt{***you use the single and the double tilde to signify "approximately". Either is fine, but please decide for one in a given paper for consistency. Please check and change according to your decision***}%%%
\DIFdelend at the solar radius, which agree excellently with the estimate found
by~\citet{2020arXiv201202169B}, $v_c(R_\odot)=244\pm 8$~km s$^{-1}$ for $R_\odot = 8.275$~kpc, or $v_c(R_\odot)=242\pm 8$~km s$^{-1}$ for our adopted value of $R_\odot$. Although this velocity is higher than the standard value (220-230 km s$^{-1}$), it should be mentioned
that according to Table 1 in \citet{sofue_2020}, our computed velocity is not an outlier (see also Section 6.2 of~\citealt{Honma_2012}).
Our solution has a an average slope $s=-4.15\pm0.015~\rm{km~s}^{-1}~kpc^{-1}$, fit for $14.5~\rm{kpc} < r < 26.5~\rm{kpc}$, which is
comparable to the corresponding value of $s=-3.93\pm0.15~\rm{km~s}^{-1}~kpc^{-1}$ measured from the rotation curve of~\citet{Jiao2023}, and agrees better than the corresponding slope of the NFW-MW model (see also \cref{fig:rotcurve} for comparison).

Using the GD-1 observables, we computed our galactocentric distance projected onto the plane $z=0$ and found that it lies inside the interval 11.5 kpc $\lesssim r \lesssim 16.4$ kpc, which is displayed as a vertical shaded (gray) band in the Galaxy RC of Fig.~\ref{fig:rotcurve}. The location of the GD-1 stream orbit corresponds to $z\in [2.6, 9.7]$ kpc and thus explores the nonsphericality of the full MW models (due to the axisymmetry of baryons and NFW).
It is noticeable that both models approximately agree in their circular velocity in the GD-1 region.
Our galactocentric distance (not projected) corresponds to the interval $13.9~\textrm{kpc} \lesssim r \lesssim 16.6~\rm{kpc}$ (subject to errors in the photometric distance, as commented in Section \ref{sec:fitting}). The theoretical orbit of the stream in the Fermionic-MW has a pericenter of $14.3$~kpc and an apocenter of $24.5$~kpc, and it is currently in its pericentric passage.

% It is interesting to display the values of the acceleration vector as a function of the position along the stream.
% In Fig.~\ref{fig:acceleration} we have computed the cylindrical components of the acceleration, $a_r$ and $a_z$,
% for both MW models, along the stream as a function of $\phi_1$ (top), $r$ (middle) and $z$ (bottom).
% \begin{figure}
%    \centering
%    \includegraphics[width=\hsize]{figures/acceleration_along_phi1.pdf}
%    \includegraphics[width=\hsize]{figures/acceleration_along_R.pdf}
%    \includegraphics[width=\hsize]{figures/acceleration_along_z.pdf}
%    \caption{Cylindrical Galactocentric componentes of the acceleration along the GD-1 stream,
%    as a function of $\phi_1$ (top), $r$ (middle) and $z$ (bottom), for both the NFW-MW and the
%    Fermionic-MW models. Cold stellar streams are excellent accelerometers.}
%    \label{fig:acceleration}
% \end{figure}
We computed the cylindrical components of the acceleration, $a_r$ and $a_z$,
for the two MW models along the stream and obtained a maximum difference of $|\Delta a_r| \lesssim 0.08 \rm{~km~s}^{-1} \rm{Myr}^{-1}$ and $|\Delta a_r| \lesssim 0.15 \rm{~km~s}^{-1} \rm{Myr}^{-1}$, respectively. This agreement supports  the idea that cold tidal streams are excellent accelerometers~\citep{Ibata_2016,2022ApJ...940...22N,2023ApJ...945L..32C}.

With respect to virial quantities, the \DIFdelbegin \textit{\DIFdel{core}}%DIFAUXCMD
\DIFdel{-}\textit{\DIFdel{halo}} %DIFAUXCMD
\DIFdelend \DIFaddbegin \DIFadd{core-halo }\DIFaddend dark matter solution has a finite virial radius $r_{\rm{DM,vir}}=27.4$ kpc and a virial mass
$M_{\rm{DM,vir}}=1.4\times10^{11} M_\odot$. The total baryon mass of our model is $M_b=0.9\times10^{11}M_\odot$, and the total virial mass therefore amounts to
$M_{\rm{vir}}=2.3\times10^{11} M_\odot$. The value of the MW total mass at 50 kpc reported in Table 3
of \citet{2014MNRAS.445.3788G} is $2.9\times10^{11} M_\odot$ with $(\sigma, 2\sigma)=(0.4,0.9)\times10^{11} M_\odot$, which means that our solution lies in the $2\sigma$ region.
The mass of the fermionic solution is constant for radii larger than
$r_{\rm{DM,vir}}$, while the mass of the model studied by \citet{2014MNRAS.445.3788G} continues to increase with radius according to their Table 3, although their estimates at large radii have relatively
high error bounds, for instance, $2\sigma=3\times10^{11} M_\odot$ for $M(200 \hbox{ kpc}) =5.6\times10^{11} M_\odot$.

The most recent MW mass estimates were obtained by
\citet{Jiao2023} and \citet{Ou2023} from \DIFdelbegin \textit{\DIFdel{Gaia}} %DIFAUXCMD
\DIFdelend \DIFaddbegin \DIFadd{Gaia }\DIFaddend DR3 data. These authors reported data that are compatible with even lower values of the MW virial mass: $1.99^{+0.09}_{-0.06}\times10^{11} M_\odot$ and $2.13^{+0.17}_{-0.12}\times10^{11} M_\odot$, respectively, in agreement with our fermionic model predictions. These MW mass estimates correspond to a sharp Keplerian decline in the MW rotation that ends at $\approx 26.5$ kpc (with an enclosed dynamical mass at this radius of $\approx 2\times 10^{11} M_\odot$, \citealp{Jiao2023}). This again agrees remarkably well with the virial radius predicted by our Fermionic-MW model of $\approx 27$ kpc.

The fitted fermionic DM model has a density in the solar neighborhood of
$\rho_{\mathrm{DM},\odot}=1.46\times10^7~M_\odot~\mathrm{kpc}^{-3}=0.55~\mathrm{GeV~cm}^{-3} c^{-2}$, which falls within the $2\sigma$ region of the estimate made by \citealt{Salucci2010} ($0.43\pm 0.21~\mathrm{GeV~cm}^{-3} c^{-2}$),
but is higher than the one obtained by~\citealt{Eilers_2019} ($0.30\pm0.03~\mathrm{GeV~cm}^{-3} c^{-2}$) or by~\citealt{Ou_2024MNRAS} ($0.447\pm0.004~\mathrm{GeV~cm}^{-3} c^{-2}$).

\subsection{Example of the S-cluster fit: Paradigmatic case of the S2 orbit}
In this section, we answer the relevant question of how well the fermionic model that fits the stream according to the procedure of \cref{sec:fitting} can fit the iconic S2 star orbit with a focus in Sagittarius~A*. Even if a good fit is expected since the core mass of the DM distribution $M_{\rm{core}}=3.5\times 10^6 M_\odot$ was fit together with the stream constraint, it is important to note that the free parameters of the Fermionic-MW model are not precisely the same as those obtained in \cite{2020A&A...641A..34B}. That is, while in \cite{2020A&A...641A..34B} the free DM model parameters were obtained to explain the S2 star geodesic together with the MW RC as given in \cite{sofue_rotation_2013}, here, the DM halo region was instead fit in order to reproduce the GD-1 stream, with somewhat different ($\beta_0$,$\theta_0$,$W_0$) values.

We therefore performed a least-squares fit following \cite{2020A&A...641A..34B} for the case of the S2 orbit as an example. As commented above, in this case, we used the Fermionic-MW DM model, which fits the GD-1 stream best, that is, $(\theta_0, \omega_0, \beta_0)= (36.0704, 27.3501, 1.2527\times10^{-5})$. We obtained excellent results. In \cref{fig:S2_fit} , we show the projected S2 orbit in the plane of the sky, while in \cref{fig:S2_fit_b}, we show the time evolution of the redshift function $z$ (which is directly related to the heliocentric radial velocity according to Equation (C.17a) in~\citealp{2020A&A...641A..34B}), $\alpha$ and $\delta$, for the best-fitting values of the \DIFdelbegin \textit{\DIFdel{osculating}} %DIFAUXCMD
\DIFdelend \DIFaddbegin \DIFadd{osculating }\DIFaddend orbital parameters. These values are given in Table~\ref{tab:S2}, along with the model-predicted value of the periapsis precession per orbital period, $\Delta \phi$, and the orbital period, $P$.

Our fitting procedure was applied in the gravitational field of two different scenarios: a Fermionic-MW DM model for $m=56~{\rm keV}~c^{-2}$ and $M_{\rm{core}}=3.5\times 10^6 M_\odot$, and a Schwarzschild BH model with a central mass of $M_{\rm BH}=4.075\times 10^6 M_\odot$.
The resulting values for the $\chi^2_{\rm{S2}}$ minimization presented in Table~\ref{tab:chi2} agree perfectly with those obtained in \cite{2020A&A...641A..34B}\footnote{Following their procedure, we have minimized $\chi^2_{\rm{S2}}$ but we have not computed the posterior distribution of the parameters, lacking the corresponding errors.}.
We used the latest publicly accessible data from \cite{2019Sci...365..664D}.

For the exemplified case of the S2 star, the models differ in the predicted value of the relativistic precession of the S2 periapsis $\Delta \phi$. This interesting relativistic effect in the case of a regular (i.e., nonsingular) DM core was compared with the publicly available astrometric data of S2 and was compared with the BH case in \cite{2022MNRAS.511L..35A}. These authors showed that higher particle masses (i.e., leading to more compact DM cores, as detailed in the section below) imply that less extended DM mass fills the S2 orbit. Thus, precession growths from retrograde to prograde as it tends to the unique value predicted by the BH model. \cite{2022MNRAS.511L..35A} showed that already for particle masses of $m = 60~{\rm keV}~c^{-2}$, which is slightly above the value considered here, $m=56~{\rm keV}~c^{-2}$, the periapsis precession is very close the one predicted by the Schwarzschild BH. Conversely, for particle masses $m\lesssim 56~{\rm keV}~c^{-2}$, the DM core is too extended in radius and produces high values of retrograde S2 periapsis precession and poorer orbit fits \citep{2020A&A...641A..34B,2022MNRAS.511L..35A}. %of $\Delta \phi \approx 12 ~{\rm arcmin\ rev}^{-1}$.
% As further detailed in the next subsection, different DM core concentrations can be achieved ranging the value of the DM particle mass, and consequently, predicting various possibilities for the precession. As shown in \cite{2022MNRAS.511L..35A}, there is a strong dependence of the core compacity with the particle mass. Increasing the DM particle mass, leads to less extended DM mass $\Delta M_{\rm DM}$ between pericenter and apocenter of the S2 orbit.
%
%
\begin{figure}
   \includegraphics[width=1.0\hsize]{figures/S2_Orbit_stream.pdf}
\caption{Modeled and observed projected orbit in the sky for a Fermionic-MW DM model in a solid (amber) line and a BH of $M_{\rm bh}=4.075\times 10^6 M_\odot$ as a dotted (green) line.}
   \label{fig:S2_fit}
\end{figure}

\begin{figure}
   \includegraphics[width=1.0\hsize]{figures/S2_Z_Ra_Dec_2.pdf}
   \caption{Redshift ($z$), right ascension~($\alpha$), and declination ($\delta$) as a function of time epoch for the same models as
   displayed in \cref{fig:S2_fit}.}
   \label{fig:S2_fit_b}
\end{figure}

\begin{table}[t]
\caption{Best-fitting \DIFdelbeginFL \textit{\DIFdelFL{osculating}} %DIFAUXCMD
\DIFdelendFL \DIFaddbeginFL \DIFaddFL{osculating }\DIFaddendFL orbital parameters of the orbit of the S2 star for two different models: a fermionic DM model with $m=56~{\rm keV}~c^{-2}$ and a core mass $M_{\rm{core}}=3.5\times 10^6 M_\odot$, and a BH of mass  $M_{\rm bh}=4.075\times 10^6 M_\odot$.}
\centering
%\footnotesize{
\begin{tabular}{lcc}
\hline
Parameter & Fermionic-MW & BH \\
\hline
$a$ [as] & 0.12507 & 0.12530 \\
$e$ & 0.8868 & 0.8861 \\
$\omega$ [$^{\circ}$] & 66.935 & 66.505 \\
$i$ [$^{\circ}$] & 134.396 & 134.440 \\
$\Omega$ [$^{\circ}$] & 228.195 & 228.046 \\
%$X_{\rm off}$ (as) & $-1.14\times10^{-5}$ & $-1.97\times 10^{-5}$ \\
%$Y_{\rm off}$ (as) & $2.63\times 10^{-3}$ & $2.29\times 10^{-3}$ \\
$P$ [yr] & 16.051 & 16.049 \\
%$t_{\rm p}$ (yr) & 2018.38 & 2018.38 \\
$\Delta \phi$ [arcmin rev$^{-1}$] & -6.04 & 11.95 \\
\hline
\end{tabular}
\label{tab:S2}
\end{table}
%

\subsection{Varying the fermion mass to reach more compact cores}
As already mentioned, we found a fermionic solution that agrees with both GD-1 data and
the geodesic motion of the best-studied S-cluster star around Sagittarius~A*, the S2 star. This S-stars constraint is also satisfied by any fermionic DM profile
with a core mass $m_\mathrm{c}\approx M_{\rm{core}}$ and more compact cores than the solution corresponding for $m=56$ keV, however. It is therefore interesting to determine how much compactness can be reached
while keeping both the S2 star and \hbox{GD-1} constraints, in the light of the new observations
of the Event Horizon Telescope, \cite{EHT_image}, where a shadow angular diameter of $48.7\pm7.0~\mu$as
has been measured. This diameter corresponds to a shadow radius of $\sim 2.46$ Schwarzschild radii assuming
a \DIFdelbegin \DIFdel{black hole }\DIFdelend \DIFaddbegin \DIFadd{BH }\DIFaddend mass of $M_{\rm{bh}}=4.075\times10^6~M_\odot$.
In order to extend the fermionic solutions to other values of the fermion mass ($m$), we used the second optimization algorithm described in Sec.~\ref{sec:optimization} for $m=100$, 200, 300, and 360 keV $c^{-2}$. For each fermion mass, we divided a given macroscopic orthohedron\footnote{The lower and upper bounds of the orthohedrons were given by
lower $(36, 27, 1.2\times10^{-5})$, $(37, 28, 5\times10^{-5})$, $(38, 29, 3.5\times10^{-4})$, $(40, 29, 1.3\times10^{-3})$ and
upper $(40, 31, 10^{-4})$, $(41, 32, 10^{-3})$, $(42, 32, 3\times10^{-3})$, $(44, 32, 4\times10^{-3})$ for $m=100$, 200, 300, and 360 keV $c^{-2}$ in $(\theta_0, \omega_0, \beta_0)$ space, respectively.}
in parameter space in $20^3=8000$ smaller orthoedrons. In each subregion, we performed an independent optimization in a parallel distributed scheme to search for the parameters that minimize $\chi^2_{\rm{full}}$ with the \texttt{NOMAD} algorithm. Afterward, we selected the global minimum by comparing the results of each distributed process.
The result is that for all the fermion masses, it is possible to find values of the other parameters in such a way that both the GD-1 stream and the core mass constraints  are respected with the same precision as in the initial ($m=56~\rm{keV}/c^2$) case. As shown in Fig.~\ref{fig:going_compact}, all the solutions have the same density profile in the halo region, while their difference is limited to the compactness of the core.
In Table~\ref{tab:chi2} we show the values of $\chi^2_{\rm{stream}}$
for all the fermion masses we studied. We also give the value of $\chi^2_{\rm{S2}}$ for
the Fermionic-MW and the BH models. All the fermionic models we analyzed are statistically indistinguishable\footnote{These fits were made for the same fixed core mass as in~\cite{2020A&A...641A..34B, 2021MNRAS.505L..64B}, but it can be shown that the value of $\chi^2_{\rm{S2}}$ corresponding to $m=360~\rm{keV}$ can be made as low as in the other cases by increasing the core mass by a few percent.}, and more data (e.g., central shadow feature, or closer or fainter S-stars to SgrA* than S-2) are needed in order to further constrain the particle mass range. Both projects are currently developed within our group.
The values of the core radii of these solutions are approximately 1097, 232, 35, 10, and 5 Schwarzschild radii for $m=56$, 100, 200, 300, and 360 keV $c^{-2}$, respectively. The last analyzed \DIFdelbegin %DIFDELCMD < \LEt{***if I may, I would recommend you use a spellchecker before you submit your next paper***} %%%
\DIFdelend value of $m=360$ keV $c^{-2}$ corresponds to a DM core very close to the last stable solution according to the stability criterion of \cite{2021MNRAS.502.4227A}, leading to a gravitational core-collapse into a BH of about $4\times 10^6 M_\odot$.

\begin{figure}
   \centering
   \includegraphics[width=\hsize]{figures/density_profiles.pdf}
   \caption{Fermionic DM density profiles with different core compacities (i.e., different $m$) fitting the GD-1 stream and the DM core mass, which agree with the S2 star data orbiting Sagittarius A*. The Schwarzschild radius is computed assuming a \DIFdelbeginFL \DIFdelFL{black hole }\DIFdelendFL \DIFaddbeginFL \DIFaddFL{BH }\DIFaddendFL mass $M_{\rm bh}=\SI{4.075E6}{\Msun}$.}
   \label{fig:going_compact}
\end{figure}

\begin{table}[t]
\caption{$\chi^2_{\rm{stream}}$ and $\chi^2_{\rm{S2}}$ values corresponding to the best-fit to the GD-1 stream and the S2 orbit, respectively.}
\centering
\small{
\begin{tabular}{lcccccc}
\hline
  $mc^2/$keV & 56 & 100 & 200 & 300 & 360 & BH \\
\hline \hline
$\chi^2_{\rm{stream}}$ & 13.528 & 13.530  & 13.575 & 13.862  & 13.836  \\
\hline
$\chi^2_{\rm{S2}}$ & 3.185 & 3.405 & 3.349 & 4.371 & 12.561 & 3.383 \\ \hline
\end{tabular}
}
\label{tab:chi2}
\end{table}
%
\section{Conclusions}
\label{sec:conclusions}
We have fitted the GD-1 stream located at about $14-15$ kpc from the Galaxy center and the S2 star orbit located at miliparsec scales in a MW potential consisting of a fermionic \DIFdelbegin \textit{\DIFdel{core}}%DIFAUXCMD
\DIFdel{-}\textit{\DIFdel{halo}} %DIFAUXCMD
\DIFdelend \DIFaddbegin \DIFadd{core-halo }\DIFaddend DM distribution
\citep{arguelles_novel_2018,2020A&A...641A..34B,2021MNRAS.505L..64B,2022MNRAS.511L..35A}, plus a fixed baryonic distribution \citep{2017A&A...598A..66P}. Remarkably, the total resulting MW mass and the virial radius of the Galaxy predicted by the fermionic DM model agree very well with the virial mass of $\approx 2\times 10^{11} M_\odot$ and the galactocentric radial range of $\approx 20-26$ kpc in which the MW rotation curve sharply drops, as recently meassured by \DIFdelbegin \textit{\DIFdel{Gaia}} %DIFAUXCMD
\DIFdelend \DIFaddbegin \DIFadd{Gaia }\DIFaddend DR3. This relatively low-mass MW is also consistent with the independent estimate given by~\cite{2014MNRAS.445.3788G}, although it is considerably lower than typical values given in the literature, for instance, ~\DIFdelbegin %DIFDELCMD < \cit{2010MNRAS.406..264W} %%%
\DIFdelend \DIFaddbegin \DIFadd{\mbox{%DIFAUXCMD
\cite{2010MNRAS.406..264W} }\hskip0pt%DIFAUXCMD
}\DIFaddend and references therein.

We obtained the free parameters of the fermionic model by fixing the fermion mass and by simultaneously fitting two astrophysical constraints: the stream observables, and a
DM core mass of $3.5\times10^6M_\odot$, the latter taken from previous fits of the S-stars cluster at
the center of the Galaxy without a central BH~\citep{2020A&A...641A..34B,2021MNRAS.505L..64B}. We could thus reproduce the polynomials fit by~\citet{Ibata_2020} with a high accuracy and for fermion masses ranging from 56 to 360~keV~$c^{-2}$, which correspond to the observed sky position, the proper motion, the photometric distance and the radial velocity of the stream.

In order to compare our results with other GD-1 fits in the literature, we also fit the initial conditions of the stream progenitor with the  axisymmetric generalization of the NFW distribution from~\citet{2019MNRAS.486.2995M}. We obtained agreement in the GD-1 phase-space track of
both the Fermionic-MW and the NFW-MW models. Additionally, we obtained agreement of the two models in the rotation curves at projected ($z=0$) galactocentric distances, $r$, corresponding to the stream observables, that is, $11.5\leq r \leq 16.4$ kpc. The average slope of the rotation curve between $14.5$ and $26.5$ kpc obtained from the Fermionic-MW model was $s=-4.18 \pm 0.02~\rm{km\, s}^{-1} kpc^{-1}$, which agrees much better with the recent observations of \cite{Jiao2023} than the NFW-MW model. We showed for the acceleration field that the two MW models perfectly agree in their acceleration vectors as a function of the position along the stream.

% relativistic fermionic King
% \footnote{[\textbf{explain why the difference w.r.t the $4\times 10^6 M_sun$ inferred from GRAVITY}]}
We found a circular velocity at the position of the Sun of $v_c(R_\odot)= 244~\rm{km~s}^{-1}$, which is in line with the value independently obtained by \citet{2019MNRAS.486.2995M}.

The fermionic DM solution has a finite radius of $r_{\rm{DM,vir}}=27.4$~kpc and a virial mass of
$M_{\rm{DM,vir}}=1.4\times10^{11}~M_\odot$, implying a total (DM plus baryons) virial mass of the Galaxy of $M_{\rm{vir}}= 2.4\times10^{11}~M_\odot$, which is at $2\sigma$ from the value reported in Table 3 of
\citet{2014MNRAS.445.3788G} for a radius of 50~kpc. The value obtained for the DM density in the solar neighborhood is $\rho_{DM,\odot}=1.46\times10^7M_\odot~\mathrm{kpc}^{-3}=0.55~\mathrm{GeV}~\mathrm{cm}^{-3} c^{-2}$, which falls inside the $2\sigma$ region of a previous estimate by \citet{Salucci2010}.

Finally, we showed that it is possible to find a one-dimensional family of solutions parameterized by the fermion mass with the same halo that fits the GD-1 stream, but with a different compactness of the central core. This always reproduces the S2 star orbit (see \cref{fig:S2_fit,fig:S2_fit_b} for the case of $m=56$ keV). For the limiting case studied here ($m=360$~keV), we obtained a core radius of $r_\mathrm{c}\approx5$~Schwarzschild radii.
A precise relativistic ray-tracing study about simulated ring-like images of the central
cores of fermionic distributions is in progress. We try to place strict constraints on the minimum
compactness needed for an agreement with the EHT observations.
%DIF <  assuming a black hole mass of $m_{\rm{bh}}=4.075\times10^6~M_\odot$.


In summary, the findings of this work not only support the idea that cold tidal streams are excellent probes of the acceleration field of the \DIFdelbegin \DIFdel{Milky Way}\DIFdelend \DIFaddbegin \DIFadd{MW}\DIFaddend , but they show that the (spherical) fermionic model is capable of fitting an independent set of observables covering three totally different Galaxy distance scales: $\sim 10^{-6}$ kpc (S-cluster), $\sim 14$ kpc (GD-1), and $\sim 30$ kpc (\DIFdelbegin \textit{\DIFdel{Gaia}} %DIFAUXCMD
\DIFdelend \DIFaddbegin \DIFadd{Gaia }\DIFaddend DR3 RC mass estimates).

%of adequately describing both the halo of the Galaxy and its central mass concentration. In fact, with a single DM model, we were capable to obtain a good fit in three different galactic distance scales: $\sim 10^{-6}$ kpc (S-cluster), $\sim 15$ kpc (GD-1) and $\sim 30$ kpc (rotation curve mass estimates).

%______________________________________________________________


\begin{acknowledgements}
    MFM would like to dedicate this work to the memory of a great friend since childhood, Gustavo Fabián Larrion (1980-2021), Q.E.P.D.
    MFM thanks Khyati Malhan for his comments about the NFW-MW model.
    MFM thanks Leandro Martínez, Ian Weaver, Joaquín Pelle and the \textsc{Julia} community for their
    great support with the language and workflow. We thank Jorge A. Rueda for his comments
    about the manuscript.
    We thank Juan Ignacio Rodriguez for his great support with hardware and
    software issues regarding the IALP server and personal computers. We thank
    Federico Bareilles and the informatic support team of the FCAGLP for their help
    during these years.
    This work used computational resources from CCAD – Universidad
    Nacional de Córdoba (\href{https://ccad.unc.edu.ar/}{https://ccad.unc.edu.ar/}),
    which are part of SNCAD – MinCyT, República Argentina.
    This work also used computational resources from the HPC center DIRAC,
    funded by Instituto de Física de Buenos Aires (UBA-CONICET) and part of SNCAD-MinCyT initiative, Argentina.
    DDC and MFM acknowledge support from CONICET (PIP2169) and from the Universidad Nacional de La Plata (G168). C.R.A. acknowledges support from CONICET, the ANPCyT (grant PICT-2018-03743), and ICRANet. VC thanks financial support from CONICET, Argentina.
    The figures presented in this work where made with the
    {\it AlgebraOfGraphics.jl}~(\href{https://aog.makie.org/dev/}{https://aog.makie.org/dev/}),
    {\it Makie.jl}~\citep{DanischKrumbiegel2021} and {\it Matplotlib}~\citep{Hunter:2007}
    packages. Some of our optimization results were initially guided by the use of the \texttt{LN\_NELDERMEAD}~\citep{NELDERMEAD,10.1093/comjnl/8.1.42,shere} algorithm, from the {\it NLopt.jl}~\citep{NLopt} package.
    In order to run {\sc Julia} in a parallel SLURM environment we made use of the {\it Distributed.jl} and \href{https://github.com/kleinhenz/SlurmClusterManager.jl}{\it SlurmClusterManager.jl} packages.
\end{acknowledgements}

% WARNING
%-------------------------------------------------------------------
% Please note that we have included the references to the file aa.dem in
% order to compile it, but we ask you to:
%
% - use BibTeX with the regular commands:
%   \bibliographystyle{aa} % style aa.bst
%   \bibliography{Yourfile} % your references Yourfile.bib
%
% - join the .bib files when you upload your source files
%-------------------------------------------------------------------
\bibliographystyle{aa} % style aa.bst
\bibliography{refs} % your references Yourfile.bib
%

\begin{appendix}
\section{Numerical solution of the system of differential equations}
\label{app:numerical}
In this section, we explain how we numerically solved the physical equations defined in Sec.~\ref{sec:fermionicDM}.
We start by defining some constants,
\begin{align}
    \rho_{\bullet}&= g\pi^{3/2} m^4 c^3 h^{-3},\\
    r_\bullet &= c/(8\pi G \rho_\bullet)^{1/2},
    % M_ \bullet &= 4\pi R^3 \rho_{\rm{rel}},
\end{align}
and introduce the following changes of variables:
\begin{align}
   \zeta &= \ln(r/r_\bullet),\\
   z(\zeta) &=\ln\psi(r(\zeta)),\\
   \tilde{\nu}(\zeta) &=\nu(r(\zeta)),\\
   \beta(\zeta)&=\frac{kT(r(\zeta))}{mc^2},\\
   \alpha(\zeta)&=\frac{\mu(r(\zeta))}{mc^2},\\
   \epsilon_c(\zeta)&=\frac{E_c(r(\zeta))}{mc^2},\\
   \epsilon(p)&=\frac{E(p)}{mc^2},
\end{align}
where
\begin{equation}
    \psi(r) = 1-\textrm{e}^{-\lambda(r)} = \frac{2G}{c^2}\frac{M(r)}{r},
    \label{mass_dm}
\end{equation}
into equations~(\ref{mass_def}), (\ref{tov}) and~(\ref{tke}), obtaining
\begin{align}
   \label{sode}
   \frac{\mathrm{d}z}{\mathrm{d}\zeta} & = -1+\mathrm{e}^{2\zeta-z}\frac{\tilde{\rho}(\zeta)}{\rho_{\bullet}},\\
   \label{sode2}
   \frac{\mathrm{d}\tilde{\nu}}{\mathrm{d}\zeta} & = \left(\mathrm{e}^{z}+\mathrm{e}^{2\zeta}\frac{\tilde{P}(\zeta)}{\rho_{\bullet}c^2}\right)(1-\mathrm{e}^{z})^{-1},\\
   \frac{1}{\beta}\frac{\mathrm{d}\beta}{\mathrm{d}\zeta}&=\frac{1}{\alpha}\frac{\mathrm{d}\alpha}{\mathrm{d}\zeta}=
   \frac{1}{\epsilon_\mathrm{c}}\frac{\mathrm{d} \epsilon_\mathrm{c}}{\mathrm{d}\zeta}=-\frac{1}{2}\frac{\mathrm{d}\tilde{\nu}}{\mathrm{d}\zeta}.\label{tke2}
\end{align}


The thermodynamical quantities, density and pressure, are given by

\begin{align}
    \tilde{\rho}(\zeta)&=\frac{4\rho_{\bullet}}{\sqrt{\pi}}\int^\infty_1\epsilon^2[\epsilon^2-1]^{1/2}\tilde{f}(\zeta,\epsilon)\mathrm{d}\epsilon,\\
   \tilde{P}(\zeta)&=\frac{4c^2\rho_{\bullet}}{3\sqrt{\pi}}\int^\infty_1[\epsilon^2-1]^{3/2}\tilde{f}(\zeta,\epsilon)\mathrm{d}\epsilon,
\end{align}
where the fermionic King distribution as a function of
$\epsilon=E/mc^2$ in the new variables is given by
\begin{equation}
\tilde{f}(\zeta,\epsilon)\equiv \frac{h^3}{g}f(r(\zeta),p(\epsilon))=
      \frac{1-\mathrm {e}^{[\epsilon-\epsilon_\mathrm{c}(\zeta)]/\beta(\zeta)}}{1+\mathrm {e}^{[\epsilon-\alpha(\zeta)]/\beta(\zeta)}}\quad\mathrm{if}\quad \epsilon \leq \epsilon_\mathrm{c}(\zeta),
\end{equation}
and $\tilde{f}(\zeta,\epsilon) = 0$ otherwise.

Equations~(\ref{tke2}) can be analytically integrated to obtain
\begin{align}
 \beta(\zeta) &= \beta_0\mathrm{e}^{-\tilde{\nu}(\zeta)/2}, \nonumber\\
 \alpha(\zeta) &= \alpha_0\mathrm{e}^{-\tilde{\nu}(\zeta)/2}, \nonumber\\
 \epsilon_c(\zeta) &= \epsilon_{c0}\mathrm{e}^{-\tilde{\nu}(\zeta)/2},
 \label{tke_solved}
\end{align}
thus transforming the original system of five differential equations, that is, ~(\ref{sode}-\ref{tke2}), into a system of just two differential equations that are to be solved numerically subject to the constraints (\ref{tke_solved}).

It is not possible to integrate these equations from $r=0$ because $\zeta(r)$ diverges at the origin.
Therefore, the following approximations for the initial conditions at a value $r_{\rm{min}}\gtrsim 0$ were used:
\begin{align}
   \nu(r_{\rm{min}}) &= \frac{1}{3}\frac{\rho_0}{\rho_\bullet}\left[\frac{r_{\rm{min}}}{r_b}\right]^2\equiv \tau, \\
   \psi(r_{\rm{min}})&= \frac{1}{3}\frac{\rho_0}{\rho_\bullet}\left[\frac{r_{\rm{min}}}{r_\bullet}\right]^2 = \tau,
\end{align}
which implies
\begin{equation}
   \frac{r_{\rm{min}}}{r_\bullet}=\sqrt{3\tau\frac{\rho_\bullet}{\rho_0}},
\end{equation}
where
$\tau\equiv 2\times10^{-15}$ and

\begin{equation}
    \rho_0\equiv \rho(0) = \frac{4\rho_{\bullet}}{\sqrt{\pi}}\int^\infty_1\epsilon^2[\epsilon^2-1]^{1/2}\tilde{f}_0(\epsilon)\mathrm{d}\epsilon,
\end{equation}

where

\begin{equation}
\tilde{f}_0(\epsilon)=
      \frac{1-\mathrm {e}^{[\epsilon-\epsilon_\mathrm{c0}]/\beta_0}}{1+\mathrm {e}^{[\epsilon-\alpha_0]/\beta_0}}\quad\mathrm{if}\quad \epsilon \leq \epsilon_\mathrm{c0},
\end{equation}
and $\tilde{f}_0(\epsilon)=0$ otherwise.

In this way, the initial conditions of our numerical system are given
by $\zeta_{\rm{min}}=\zeta(r_{\rm{min}})$,
$\tilde{\nu}_{\rm{min}}=\tau$, and $z_{\rm{min}}=\ln{\tau}$, and the system parameters to be varied are $m$, $\beta_0$, $\alpha_0$, and $\epsilon_{c0}$.
In agreement with Eqs.~(\ref{rar_params}), we used as parameters the following normalized quantities: $m$, $\beta_0$, $\theta_0=(\alpha_0-1)/\beta_0$, and $W_0=(\epsilon_{\mathrm{c}0}-1)/\beta_0$, or $\omega_0=W_0-\theta_0$ instead of $W_0$ in some cases.

Equations~(\ref{sode}) and (\ref{sode2}) were solved using
a {\sc Python}~\citep{van1995python} script\footnote{\texttt{model\_def.py}}
that makes use of the {\it NumPy}~\citep{2020SciPy-NMeth} and {\it SciPy}~\citep{harris2020array} libraries, under the \texttt{LSODA} algorithm as solver.  We used relative and absolute tolerance parameters given by \texttt{rtol}$=5\times10^{-14}$ and \texttt{atol}$=0$, respectively.

After obtaining the numerical solution, since the right sides of Eqns.~(\ref{sode}) and (\ref{sode2}) do not depend on the metric, but only on its radial derivative, we can add
a constant $\tilde{\nu}_0$ to the solution in order to satisfy the condition of continuity with the Schwarzschild metric at the border of the fermion distribution, obtaining
\begin{equation}
\tilde{\nu}_0 = 2\ln\left(\frac{\beta_\mathrm{b}}{\beta_0}\sqrt{1-\psi_\mathrm{b}}\right),
\end{equation}
where $\psi_\mathrm{b}$ and $\beta_\mathrm{b}$ are quantities evaluated at the border, that is, when
$\rho(r_\mathrm{b})\rightarrow 0$.\footnote{The border was defined numerically as the radius in which
the density decays to $\rho_b=10^{-10}~M_\odot~\rm{pc}^{-3}$.}


% We found that among all the solvers available in the latter package the \texttt{LSODA} algorithm was the only one with which we could obtain a solution.
\end{appendix}

\end{document}






